% !TEX root = main.tex

\chapter{Adversarially Robust Property-Preserving Hashing}\label{chap:PPH}
This chapter describes the new cryptographic notion: adversarially robust Property-Preserving Hash Functions (PPHs). Property-preserving hashing (without robustness) is a method of compressing a large input $\vecx$ into a short hash $h(\vecx)$ in such a way that given $h(\vecx)$ and $h(\vecy)$, one can compute a property $P(\vecx,\vecy)$ of the original inputs. The idea of property-preserving hash functions underlies sketching, compressed sensing and locality-sensitive hashing.

Property-preserving hash functions are usually probabilistic: they use the random choice of a hash function from a family to achieve compression, and as a consequence, err on some inputs. Traditionally, the notion of correctness for these hash functions requires that for every two inputs $\vecx$ and $\vecy$, the probability that $h(\vecx)$ and $h(\vecy)$ mislead us into a wrong prediction of $P(\vecx,\vecy)$ is negligible. As observed in many recent works (incl. Mironov, Naor and Segev, STOC 2008; Hardt and Woodruff, STOC 2013; Naor and Yogev, CRYPTO 2015), such a correctness guarantee assumes that the adversary (who produces the offending inputs) has no information about the hash function, and is too weak in many scenarios.

This chapters covers the study of {\em adversarial robustness} for property-preserving hash functions, providing definitions, deriving broad lower bounds due to a simple connection with communication complexity, and showing the necessity of computational assumptions to construct such functions. Our main positive results are two candidate constructions of property-preserving hash functions (achieving different parameters) for the (promise) gap-Hamming property which checks if $\vecx$ and $\vecy$ are ``too far'' or ``too close''. Our first construction relies on generic collision-resistant hash functions, and our second on a variant of the syndrome decoding assumption on low-density parity check codes.

This chapter is based on \cite{BLV19} and unpublished work with Cyrus Rashtchian.

\section{Overview}\label{sec:pph-intro}
\section{Introduction}

\xxx{Story: fine grained crypto needs all this stuff. }

Fine-grained complexity has helped explain the hardness of a wide variety of problems. 

The idea of fine-grained one way functions were introduced by Ball et.al. \cite{avgCaseFineGrained}. 

Fine-grained cryptography ideally would have all of the following objects listed bellow. Those in bold we have created. Those italicized have been solved previously. 

\begin{itemize}
	\item Plausibly average case hardness assumptions on which to base cryptographic constructions. 
	\begin{itemize}
		\item \textit{Worst case to average case reductions to show hardness for problems.} \cite{avgCaseFineGrained}
		\item Showing problems hard from satisfiability. 
		\item \textbf{Using plausible average case hypotheses. }
		\item Assumptions that have a fine-grained implications for quantum computers as well. 
	\end{itemize}
	\item  ``Symmetric'' Fine-grained Cryptography 
		\begin{itemize}
		\item \textbf{Fine-grained one way functions (FGOWFs)}
		\item \textbf{FGOWFs with hardcore bits}
		\item Fine-grained pseudo-random generators (FGPRGs)
		\end{itemize}
	\item ``Asymmetric'' Fine-grained Cryptography 
		\begin{itemize}
		\item Fine-grained trap-door functions
		\item \textbf{Fine-grained key transfer}
		\item Fine-grained public key cryptography
		\item Fine-grained identity based encryption
		\end{itemize}
	\item Advanced constructions
		\begin{itemize}
		\item Oblivious transfer
		\item Homomorphic encryption
		\end{itemize}
\end{itemize}

%\section{Preliminaries for PPH}
%\input{pph/preliminaries.tex}

\section{Defining Property-Preserving Hash Functions}
\label{sec:pph-definitions}\label{sec:pph-defs}
\input{pph/definitions.tex}
\input{pph/multi-input-predicates.tex}

%\input{pph/two-to-single-input-proof.tex} % Relates to multi-input
%\input{pph/total-predicate-nonrobust-to-EO.tex} %Relationships between definitions

\section{Property Preserving Hashing and Communication Complexity}\label{sec:owc-relations}
\input{pph/owc-relations.tex}
%\input{pph/owc-to-pph-lb-proof.tex}
%\input{pph/owc-proofs.tex}

\section{A Gap-Hamming PPH from Collision Resistance}\label{sec:construction1}
\input{pph/collision-resistance-construction.tex}
%\input{pph/construction-proofs.tex}

\section{A Gap-Hamming PPH from Sparse Short Vectors}
\label{sec:ham}
\input{pph/gap-hamming-construction.tex}

\section{Necessity of Cryptographic Assumptions}\label{sec:lb}
\input{pph/other-assumptions.tex}
%\input{pph/eq-implies-owf.tex}

% Appendix stuff
%\input{pph/omitted-proofs.tex} %---- Moved all these inside!