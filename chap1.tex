%% This is an example first chapter.  You should put chapter/appendix that you
%% write into a separate file, and add a line \include{yourfilename} to
%% main.tex, where `yourfilename.tex' is the name of the chapter/appendix file.
%% You can process specific files by typing their names in at the 
%% \files=
%% prompt when you run the file main.tex through LaTeX.
\chapter{Introduction}

One of the core goals of cryptography is to be able to offer security and privacy without sacrificing functionality.
To do this, cryptographers define notions of both correctness and security. We can then build systems that we prove satisfy both of these definitions. Correctness states that the system provides the functionality we want, while security is essentially the notion that we can effectively hide information from an adversary. Depending on the notion of security, this adversary may be all powerful (information-theoretic or computationally unbounded), may run only in polynomial time (computationally bounded), or even bounded by a specific polynomial runtime like $O(n^2)$ (a fine-grained adversary). This thesis will focus on realizing three different functionalities, each one hiding different information, and three different notions of security.

%THC
In the first part of this thesis, we discuss results in the area of Topology Hiding Computation (\THC). \THC~is a generalization of secure multiparty computation. In this model, parties are in an incomplete but connected communication network, each party with its own private inputs. The functionality these parties want to realize is evaluating some function (computation) over all of their inputs. On the security and privacy side, these parties want to hide both their private inputs and who their neighbors are.
So, the goal is for these parties to run a protocol, communicating only with their neighbors, so that by the end of the protocol, they learn the output of the function on their inputs and \emph{nothing else}, including information about what the communication network looks like other than their own neighborhood.

It turns out that \THC~is a difficult notion to realize, mired in impossibility results.
We circumvent two impossibility results in \THC~in two different ways. First, there is an impossibility result for \THC~stating that it is impossible to design \THC~against adversaries that can ``turn off'' parties during the protocol \cite{MOR15}. In essence, an adversary that has this power can always learn something about the graph. So, instead of designing a protocol that would attempt to leak no information, we designed a protocol to limit the amount of information leaked as much as possible. We ran into similar impossibility results when considering a model more similar to the real-world: channels between parties now have unknown delays on them, and an adversary may be able to control the delay. We had to find a model that mirrored this real-world complication, and this was also not impossible to achieve.

% PPH: information theoretic adversaries
In the second part of the thesis, we focus on Adversarially Robust Property Preserving Hashes (PPH).
PPHs are a generalization of collision resistant hash functions (CRHFs). Recall that a CRHF is a family of hash functions $\cH = \{ h:\{0,1\}^n \to \{0,1\}^m \}$ that is compressing ($m < n$), and has the property that no Probabilistic Polynomial-Time (PPT) adversary given $h$ can find two inputs $x_1 \neq x_2$ such that $h(x_1) = h(x_2)$ except with negligible probability. Looking at CRHFs from a slightly different perspective, their functionality is to preserve the equality predicate between two inputs while compressing them, preventing an adversary from coming up with inputs where the equality predicate is not preserved  (even though these inputs exist, they are hard to find). A PPH is also a compressing function that preserves some other property. For example, we focus on the property of ``gap-hamming'' distance: if two inputs are close in hamming distance, then their hashes are also close, and if two inputs are far, their hashes are also far. Note that this example is of a promise predicate since we do not care about inputs that are neither close nor far.

Property preserving hashes were a new cryptographic primitive that we designed, and so it was important to understand what was possible and what was not in our new model. We found strong connections between One-Way Communication Complexity (OWC complexity) and our primitive. Fortunately for us, OWC is a rich, well-studied area of complexity. In many cases, we could not directly port OWC results to our setting, but we could use their proof techniques and get lower bounds. Once we had these lower bounds, we had a much better sense of what was and was not possible, and could construct PPHs more easily.

% FGC: functionality and hiding
In the final chapter, for a different perspective, we design a public-key cryptography functionality against weak adversaries. This functionality allows for a party to publish a public key while they keep the secret key, and any other party to use that public key to encrypt a message that only the party with the secret key can decrypt, \emph{hiding} that message from any eavesdroppers. Unlike in standard models for cryptography, however, our eavesdroppers much less powerful: their runtime is bounded by an explicit polynomial instead of ``probabilistic polynomial-time'', e.g. can only run in time $O(n^{100})$. This, of course, is only interesting as a cryptographic notion if the adversary has the same or more time to run than the honest parties. So, we get a notion of cryptography we call \emph{fine-grained}: for examples, honest parties might only need to run in $O(n^2)$ time, while any adversary running in time $O(n^{100})$ time still cannot glean any useful information with some probability. Motivating the study of this cryptography is the fact that it does not rely on any of the normal cryptographic assumptions, including $P \neq NP$, $BPP \neq NP$, or even the assumption that one-way functions exist.

However, due to its fine-grained nature many standard cryptographic reductions (like Goldreich-Levin hardcore bits \cite{hardCoreBitsAndXorLemmaFromGL}) no longer work. And so, in this thesis we demonstrate variations that can work in our setting, including a notion of fine-grained one-way functions, fine-grained hardcore-bits, a fine-grained key exchange, and finally fine-grained public-key cryptography.

\section{Results}
In this thesis, we explore these three different notions of hiding while preserving a functionality:
%will explore what to do when we hit one of these impossibility results described above, therefore going beyond impossibility. We explore this from three different perspectives:
\begin{enumerate}
	\item hide private inputs and network topology while preserving computation,
	%Weaken the adversarial model so that the problem we are trying to solve is no longer impossible, and so that the model still provides a meaningful notion of security.
	\item hide as much information as possible while preserving a property between inputs,
	%Use the impossibility results to direct feasibility results and constructions.
	\item and hide messages from a weak adversary while preserving the communication and fine-grained runtime.
\end{enumerate}

\paragraph{Topology-Hiding Computation}
To address the first perspective, we explore the realm of topology-hiding computation (\THC). \THC~is a generalization of secure multiparty computation (MPC), where we hide not only each party's input, but also the communication graph; parties know who their neighbors are, but learn nothing else about the structure of the graph. In 2017, we showed that \THC~is possible against a very weak adversary \cite{ALM17}, but there were still many open questions surrounding the nature of \THC.
\begin{itemize}
	\item \textbf{Fail-stop Adversaries.} Fail-stop adversaries can turn off parties during the protocol. As shown by Moran, Orlav, and Richelson in 2015 \cite{MOR15}, it is impossible to achieve perfect topology hiding when giving the adversary this power. The next question is how to get around such an impossibility while still providing meaningful privacy and security. Since the adversary was going to learn something in any protocol we could devise, we decided to use a model that would incorporate this leakage.
	
	Our results here were a definition of security that included access to a ``leakage oracle'' and a protocol that used this oracle in the security proof. This oracle would spit out yes or no answers to specific queries, leaking one bit of information at a time. In the security reduction of our protocol, a simulator would only need to call this oracle at most once, leaking at most one bit of information to an adversary. Concretely, we could bound the probability that the simulator would have to call the leakage oracle, meaning that we would leak only a polynomial-fraction of a bit. Our protocol ran in $\~O(\kappa n^4 / \rho)$ rounds, where $\kappa$ is the security parameter, $n$ is the number of parties, and $\rho$ is the probability that we leak a bit.
	
	These results are detailed in our paper at TCC 2018 \cite{LLMMMT18}.
	
	\item \textbf{Probabilistic Delay Model.} This model was the result of first trying to build \THC~in a purely asynchronous setting, only to prove that it was impossible to do so. So, to get around this without just reducing ourselves to the fail-stop case, we designed a new model that more closely resembled an asynchronous, real-world setting.
	
	More formally, each edge between parties in the Probabilistic Delay Model has its own distribution over delays. Each time a party sends a message to another party, that delay is sampled \emph{independently}. This simulates real world traffic having different delays as it gets sent from one server to another.
	
	Our results also included two new constructions in this model. The first is a construction that works under standard cryptographic assumptions, but only on trees, cycles, or graphs with small circumference. The second is a construction that requires secure hardware boxes/tokens, but works for arbitrary communication graphs. Although this is a very powerful assumption, there are still many challenges involved since an adversarial party can query this hardware at any point with any input.
	
	These results are currently in submission.
\end{itemize}

\paragraph{Adversarially Robust Property Preserving Hashes and One-Way Communication Lower Bounds}
Addressing the second perspective, we look at a new cryptographic primitive: adversarially robust Property Preserving Hashes (\PPH). The inspiration here is two-fold. First, collision-resistant hash functions (CRHFs) are a staple in cryptography: no PPT adversary can produce an $x_1 \neq x_2$ such that $h(x_1) = h(x_2)$ except with negligible probability over $h$ sampled from the CRHF family and coins of the adversary, and hence one can preserve the ``equality'' predicate on compressed inputs even against adversarially chosen inputs. We want to compress inputs while maintaining some property other than equality between them, even in the presence of adversarially chosen inputs. The properties we considered were those that already had non-robust constructions, in particular locality-sensitive hashing (LSH) \cite{IndykM98}. An LSH family is a hash function family $\cH = \{ h:\{0,1\}^n \to \{0,1\}^m \}$ that compresses inputs ($m < n$) and is \emph{mostly} correct. For example, say we were preserving a gap-$\ell_p$ norm for a gap between $r$ and $cr$ for some constant $c > 1$ (note that in our case, gap-hamming is equivalent to gap-$\ell_0$ norm). Then, we have a threshold $\tau$ so that for any pair $x_1, x_2 \in \{0,1\}$
\begin{itemize}
	\item if $||x_1 - x_2||_p < r$, $||h(x_1) - h(x_2)||_p < \tau$, and
	\item if $||x_1 - x_2||_p > cr$ then $||h(x_1) - h(x_2)||_p \ge \tau$
\end{itemize}
with high probability over our choice of $h$ sampled from $\cH$.

We can use almost the same language to define PPHs for some property $P:\{0,1\}^n \times \{0,1\}^n \to \{0,1\}$, except our probability will be negligible and taken over both our sampled hash function and the coins of a PPT adversary. We also generalize the predicate evaluation: instead of using the same predicate on hashed values, we can use an ``evaluation function'' called $\eval$. So, even in the presence of (PPT) adversarially chosen values of $x_1$ and $x_2$,
\begin{itemize}
	\item if $P(x_1, x_2) = 0$, then $\eval(h(x_1), h(x_2)) = 0$, and
	\item if $P(x_1, x_2) = 1$, then $\eval(h(x_1), h(x_2)) = 1$
\end{itemize}
with all but negligible probability.
In other words, for any PPT adversary given $h$ sampled from $\cH$, the probability that the adversary produces $x_1$ and $x_2$ such that $P(x_1,x_2) \neq \eval(h(x_1), h(x_2))$ is negligible.

The link between robust PPH and LSH was very clear, but less obvious was the link between PPH and one-way communication (OWC): in essence, compressing an input as much as possible for a OWC protocol is akin to compressing an input as much as possible while preserving some property of that input as we would want to do for a hash function. These connections led to the following results, which will be included in the thesis.
\begin{itemize}
	\item OWC is a rich field with a lot of work detailing lower bounds on the complexity of certain predicates. These lower bounds \emph{almost} directly translated to impossibility results for PPHs. However, we needed to be careful because OWC lower bounds dealt with constant error, where as cryptographers, we care about negligible error. We characterized these OWC lower bounds and how they mapped to PPH lower bounds. We showed that a class of predicates we call ``reconstructing predicates'' could not have PPHs. This class includes useful predicates like GreaterThan, Index, and ExactHamming.
	\item With these lower bounds and previous results from LSH in mind, we turned to constructing a PPH for Gap-Hamming: the promise predicate where we can distinguish between pairs of points that are $(1-\epsilon)d$ \emph{close} in hamming distance or $(1 + \epsilon)d$ \emph{far}. For all pairs within the gap, we ``don't care'' how our PPH behaves. We build two constructions for this predicate, each with different drawbacks and benefits. Our first construction relies only on collision-resistant hashing. Our second uses a new assumption related to the hardness of syndrome decoding \cite{AHIKV17}.
	\item Rounding out this paper and its myriad of definitions, we demonstrate that even for very simple predicates, we require collision resistance, and even if we weaken our main definition of a robust PPH, we still need one-way functions.
\end{itemize}
These results were published in ITCS 2019 \cite{BLV19}.

In unpublished follow-up research with Cyrus Rashtchian, we explored the intricacies of the CRHF method for constructing gap-hamming PPHs. We fleshed out the relationship between $d$, the ``center'' of our gap, and the size of the hash function output. We also discussed how to use this kind of construction to get a gap-$\ell_1$-norm PPH and found that with standard methods of transforming $\ell_1$ into $\ell_0$, it could not work. %TODO: publish? maybe?

\paragraph{Fine-Grained Cryptography and Barriers}
Addressing the final perspective, this thesis will discuss our work in Fine-Grained cryptography. With my coauthors Lincoln and Vassilevska Williams, we built the first fine-grained key exchange built from fine-grained assumptions. The premise was to explore what cryptography could be like in ``Pessiland,'' a world in which there are no cryptographic one-way functions (which also implies no public key cryptography). We looked at this through the lens of fine-grained complexity, using both assumptions and reduction techniques from that field.

While designing cryptography for this setting, we came across multiple barriers. The first was that some fine-grained problems would not lend themselves to build cryptography without refuting NSETH \cite{CarmosinoGIMPS16}. So, we would need to use a fine-grained problem that was not associated with that barrier. Another form of barrier we ran into was worst-case to average-case reductions for problems that would make for good cryptography. Even now, the only worst-case to average-case reductions for fine-grained problems are for counting versions of these problems. Unfortunately, counting-style problems do not lend themselves well to building cryptography, as there are no longer small enough witnesses. Finally, many standard cryptographic reductions no longer work in a fine-grained world, since they take a non-trivial polynomial amount of time, and so a challenge we faced was to ensure all of our reductions were fine-grained and build different types of primitives based off of these restricted reductions.

Despite these difficulties, we achieved the following results:
\begin{itemize}
	\item Assuming that an average-case version of Zero-$k$-Clique requires $n^{k - o(1)}$ time, we constructed a non-interactive key exchange that required honest parties to run in $O(N)$ time and an adversary to run in time $\tilde\Omega(N^{1.5 - \epsilon})$, where $\epsilon$ decreases with respect to $k$.\footnote{The tilde in $\tilde \Omega$ ignores any \emph{subpolynomial} factors.}
	\item Generalize the properties of Zero-$k$-Clique to construct this key exchange: plantable, list-hard, and splittable.
	\item Define and construct fine-grained hard-core bits.
\end{itemize}
This work was published in CRYPTO 2019 \cite{LLVW19}.

In unpublished follow-up work, we also show how to use another property of Zero-$k$-Clique to construct a key exchange where the adversary now needs $\Omega(N^{2 - \epsilon})$ time. This $N^2$ gap is the best we are able to do since we base these constructions on Merkle puzzles \cite{optimalMerklePuzzles}.

\section{Related Works}
%TODO
\subsection{Related Work to Topology Hiding Computation}
Hiding the network topology is more of a concern in the literature on \emph{anonymous communication} \cite{ACM:Cha81,ACM:RR98,IEEE:SGR97}. Here, the goal is to hide the identity of the sender and receiver in a message transmission. A classical technique to achieve anonymity is the so-called mix-net technique, introduced by Chaum \cite{ACM:Cha81}. Here, \emph{mix} servers are used as proxies which shuffle messages sent between peers to disable an eavesdropper from following a message's path. The onion routing technique \cite{IEEE:SGR97,ACM:RR98} is perhaps the most known instantiation of the mix-technique. %[] 
Another anonymity technique known as \emph{Dining Cryptographers networks}, in short DC-nets, was introduced in \cite{JC:Chaum88} (see also \cite{EC:BosDen89,EC:GolJue04}). 

Existing constructions to achieve \emph{topology-hiding communication} over an incomplete network focus mainly on the cryptographic setting for passive corruptions. The first protocol was given in \cite{TCC:MorOrlRic15}. Here, the authors provide a feasibility result for a broadcast protocol secure against static, passive corruptions. At a very high level, \cite{TCC:MorOrlRic15} uses a series of nested multi-party computations, in which each node is emulated by a secure computation of its neighbor. This emulation then extends to the entire graph recursively. In \cite{C:HMTZ16}, the authors improve this result and provide a construction that makes only black-box use of encryption and where the security is based on the DDH assumption. However, both results are feasible only for graphs with logarithmic diameter. 
Topology hiding communication for certain classes of graphs with large diameter was described in \cite{EC:AkaMor17}. This result was finally extended to allow for arbitrary (connected) graphs in \cite{C:AkaLaVMor17}.

The fail-stop setting was first considered in \cite{TCC:MorOrlRic15} where the authors give a construction for a fail-stop adversary with a very limited corruption pattern: the adversary is not allowed to corrupt any complete neighborhood of a party and is also not allowed an abort pattern that disconnects the graph. In this work, the authors also prove that topology-hiding communication without leakage is impossible if the adversary disconnects the graph. A more recent work, \cite{BBMM18}, provides a construction for a fail-stop adversary with one bit/non-negligible leakage but requires that parties have access to secure hardware modules which are initialized with correlated, pre-shared keys.

In the information-theoretic setting, the main result is negative \cite{HJ07}: any MPC protocol in the information-theoretic setting inherently leaks information about the network graph. They also show that if the routing table is leaked, one can construct an MPC protocol which leaks no additional information.

%TODO: Combine this with above!!!
Topology-hiding computation was introduced by Moran et al. in \cite{TCC:MorOrlRic15}. The authors propose a broadcast protocol tolerating any number of passive corruptions. The construction uses a series of nested multi-party computations, in which each node is emulated by its neighbors. This broadcast protocol can then be used to achieve topology-hiding MPC using standard techniques to transform broadcast channels into secure point-to-point channels. In \cite{HMTZ16}, the authors provide a more efficient construction based on the DDH assumption. However, both results are only feasible for graphs with logarithmic diameter. Topology-hiding communication for certain classes of graphs with large diameter was described in \cite{EC:AkaMor17}. This result was finally extended to arbitrary (connected) graphs in \cite{ALM17}. These results were extended to the fail-stop setting in \cite{BBMM18} based on stateless secure hardware, and \cite{LLMMMT18} based on standard assumptions.
%
All of the results mentioned above are in the cryptographic setting. Moreover, all results are stated in the synchronous communication model with known upper bounds on the delays.

In the information-theoretic setting, the main result is negative \cite{HJ07}: any topology-hiding MPC protocol inherently leaks information about the network graph. This work also shows that if the routing table is leaked, one can construct an MPC protocol which leaks no additional information.

\subsection{Related Work to Property Preserving Hashing}

\subsection{Related Work to Fine-Grained Cryptography}

\section{Outline of Thesis}
%TODO
TODO
