% -*-latex-*-
% 
% For questions, comments, concerns or complaints:
% thesis@mit.edu
% 
%
% $Log: cover.tex,v $
% Revision 1.8  2008/05/13 15:02:15  jdreed
% Degree month is June, not May.  Added note about prevdegrees.
% Arthur Smith's title updated
%
% Revision 1.7  2001/02/08 18:53:16  boojum
% changed some \newpages to \cleardoublepages
%
% Revision 1.6  1999/10/21 14:49:31  boojum
% changed comment referring to documentstyle
%
% Revision 1.5  1999/10/21 14:39:04  boojum
% *** empty log message ***
%
% Revision 1.4  1997/04/18  17:54:10  othomas
% added page numbers on abstract and cover, and made 1 abstract
% page the default rather than 2.  (anne hunter tells me this
% is the new institute standard.)
%
% Revision 1.4  1997/04/18  17:54:10  othomas
% added page numbers on abstract and cover, and made 1 abstract
% page the default rather than 2.  (anne hunter tells me this
% is the new institute standard.)
%
% Revision 1.3  93/05/17  17:06:29  starflt
% Added acknowledgements section (suggested by tompalka)
% 
% Revision 1.2  92/04/22  13:13:13  epeisach
% Fixes for 1991 course 6 requirements
% Phrase "and to grant others the right to do so" has been added to 
% permission clause
% Second copy of abstract is not counted as separate pages so numbering works
% out
% 
% Revision 1.1  92/04/22  13:08:20  epeisach

% NOTE:
% These templates make an effort to conform to the MIT Thesis specifications,
% however the specifications can change.  We recommend that you verify the
% layout of your title page with your thesis advisor and/or the MIT 
% Libraries before printing your final copy.
\title{Relationships Between Functionality, Security, and Privacy}

\author{Rio LaVigne}
% If you wish to list your previous degrees on the cover page, use the 
% previous degrees command:
%       \prevdegrees{A.A., Harvard University (1985)}
% You can use the \\ command to list multiple previous degrees
%       \prevdegrees{B.S., University of California (1978) \\
%                    S.M., Massachusetts Institute of Technology (1981)}
\department{Department of Electrical Engineering and Computer Science}

% If the thesis is for two degrees simultaneously, list them both
% separated by \and like this:
% \degree{Doctor of Philosophy \and Master of Science}
\degree{Doctor of Philosophy}

% As of the 2007-08 academic year, valid degree months are September, 
% February, or June.  The default is June.
\degreemonth{May}
\degreeyear{2020}
\thesisdate{\today}

%% By default, the thesis will be copyrighted to MIT.  If you need to copyright
%% the thesis to yourself, just specify the `vi' documentclass option.  If for
%% some reason you want to exactly specify the copyright notice text, you can
%% use the \copyrightnoticetext command.  
%\copyrightnoticetext{\copyright IBM, 1990.  Do not open till Xmas.}

% If there is more than one supervisor, use the \supervisor command
% once for each.
\supervisor{Vinod Vaikuntanathan}{Associate Professor of Electrical Engineering and Computer Science at MIT}

% This is the department committee chairman, not the thesis committee
% chairman.  You should replace this with your Department's Committee
% Chairman.
\chairman{Leslie A. Kolodziejski}{Professor of Electrical Engineering and Computer Science\\
Chair, Department Committee on Graduate Students}

% Make the titlepage based on the above information.  If you need
% something special and can't use the standard form, you can specify
% the exact text of the titlepage yourself.  Put it in a titlepage
% environment and leave blank lines where you want vertical space.
% The spaces will be adjusted to fill the entire page.  The dotted
% lines for the signatures are made with the \signature command.
\maketitle

% The abstractpage environment sets up everything on the page except
% the text itself.  The title and other header material are put at the
% top of the page, and the supervisors are listed at the bottom.  A
% new page is begun both before and after.  Of course, an abstract may
% be more than one page itself.  If you need more control over the
% format of the page, you can use the abstract environment, which puts
% the word "Abstract" at the beginning and single spaces its text.

%% You can either \input (*not* \include) your abstract file, or you can put
%% the text of the abstract directly between the \begin{abstractpage} and
%% \end{abstractpage} commands.

% First copy: start a new page, and save the page number.
\cleardoublepage
% Uncomment the next line if you do NOT want a page number on your
% abstract and acknowledgments pages.
% \pagestyle{empty}
\setcounter{savepage}{\thepage}
\begin{abstractpage}
\begin{abstract}

Cryptography is largely based on unproven assumptions, which, while believable, might fail. Notably if $P = NP$, or if we live in Pessiland, then all current cryptographic assumptions will be broken. A compelling question is if any interesting cryptography might exist in Pessiland.
	
A natural approach to tackle this question is to base cryptography on an assumption from fine-grained complexity.
Ball, Rosen, Sabin, and Vasudevan [BRSV'17] attempted this, starting from popular hardness assumptions, such as the Orthogonal Vectors (OV) Conjecture. They obtained problems that are hard on average, assuming that OV and other problems are hard in the worst case. They obtained proofs of work, and hoped to use their average-case hard problems to build a fine-grained one-way function.
Unfortunately, they proved that constructing one using their approach would violate a popular hardness hypothesis. This motivates the search for other fine-grained average-case hard problems.
	
The main goal of this paper is to identify sufficient properties for a fine-grained average-case assumption that imply cryptographic primitives such as fine-grained public key cryptography (PKC).
Our main contribution is a novel construction of a cryptographic key exchange, 
together with the definition of a small number of relatively weak structural properties, such that if a computational problem satisfies them, our key exchange has provable fine-grained security guarantees, based on the hardness of this problem. We then show that a natural and plausible average-case assumption for the key problem Zero-$k$-Clique from fine-grained complexity satisfies our properties. We also develop fine-grained one-way functions and hardcore bits even under these weaker assumptions.

	
Where previous works had to assume random oracles or the existence of strong one-way functions to get a key-exchange computable in $O(n)$ time secure against $O(n^2)$ adversaries (see [Merkle'78] and [BGI'08]), our assumptions seem much weaker. Our key exchange has a similar gap between the computation of the honest party and the adversary as prior work, while being non-interactive, implying fine-grained PKC.
	

\end{abstract}

%		In 2016 Degwekar, Vaikuntanathan, and Vasudevan [DVV'16] explored a circuit-complexity-based approach, bounding computation power of adversaries by NC$^0$ or AC$^0$ for both symmetric-key and public-key primitives. While in 2017

	%The last few years have seen a resurgence in interest in fine-grained cryptography, which had its origins in fine-grained key exchanges from the 1970s (see [Merkle'78] and [BGI'08]). 
%There are two main motivations. The first is that all known c
 % and were able to prove that getting their approach to work might be hard.
 
 	%This motivates constructing cryptographic primitives from a variety of {\em unrelated} yet plausible complexity assumptions. %Ideally these assumptions would be simultaneously consistent with $P=NP$ and non-algebraic. %makes sense - if one assumption fails, another might prevail. 
 %It is thus worthwhile to build primitives from combinatorial assumptions in \emph{polynomial time} that are markedly different from the typical exponential algebraic hardness hypotheses in cryptography. %A second important goal 
 %is the search for practically efficient protocols. %Striving for security against all polynomial time adversaries often carries with it the cost of higher (though polynomial) runtime for the honest parties. Practically speaking, it makes sense to weaken the security requirement: have very efficient, hopefully linear time, honest parties, and a large but fixed polynomial cost for the adversary.

%If we make an even more believable conjecture that Zero $k$-Clique, or even the related $k$-SUM problem, require super-linear time to solve on average, we build fine-grained one-way functions and fine-grained hard-core bits.
	%
%Our contribution is to start building fine-grained one-way functions and key exchanges from fine-grained average-case problems, like $k$-sum and zero $k$-clique, as well as explore the implications. First, we characterize which problems with which properties can be used to build these one-way functions, and we show that relatively weak assumptions about fine-grained problems can yield these one-way functions; these assumptions are only that it takes super-linear time for an adversary to solve such a problem. We also show that there exist fine-grained versions of hardcore bits. 
%
%Second, we provide a novel construction for a fine-grained key exchange based on a fine-grained assumption. Where previous works had to assume random oracles or the existence of strong one-way functions to get a key-exchange computable in $n$ time secure against $O(n^2)$ adversaries, such as Merkle and Biham et al respectively, we make a much weaker assumption: zero $k$-clique is requires $\tilde \Omega(n^k)$ time to solve on average. This key exchange, like Merkle and Biham's constructions, will be non-interactive, and therefore will imply fine-grained public key encryption.
\end{abstractpage}

% Additional copy: start a new page, and reset the page number.  This way,
% the second copy of the abstract is not counted as separate pages.
% Uncomment the next 6 lines if you need two copies of the abstract
% page.
% \setcounter{page}{\thesavepage}
% \begin{abstractpage}
% \begin{abstract}

Cryptography is largely based on unproven assumptions, which, while believable, might fail. Notably if $P = NP$, or if we live in Pessiland, then all current cryptographic assumptions will be broken. A compelling question is if any interesting cryptography might exist in Pessiland.
	
A natural approach to tackle this question is to base cryptography on an assumption from fine-grained complexity.
Ball, Rosen, Sabin, and Vasudevan [BRSV'17] attempted this, starting from popular hardness assumptions, such as the Orthogonal Vectors (OV) Conjecture. They obtained problems that are hard on average, assuming that OV and other problems are hard in the worst case. They obtained proofs of work, and hoped to use their average-case hard problems to build a fine-grained one-way function.
Unfortunately, they proved that constructing one using their approach would violate a popular hardness hypothesis. This motivates the search for other fine-grained average-case hard problems.
	
The main goal of this paper is to identify sufficient properties for a fine-grained average-case assumption that imply cryptographic primitives such as fine-grained public key cryptography (PKC).
Our main contribution is a novel construction of a cryptographic key exchange, 
together with the definition of a small number of relatively weak structural properties, such that if a computational problem satisfies them, our key exchange has provable fine-grained security guarantees, based on the hardness of this problem. We then show that a natural and plausible average-case assumption for the key problem Zero-$k$-Clique from fine-grained complexity satisfies our properties. We also develop fine-grained one-way functions and hardcore bits even under these weaker assumptions.

	
Where previous works had to assume random oracles or the existence of strong one-way functions to get a key-exchange computable in $O(n)$ time secure against $O(n^2)$ adversaries (see [Merkle'78] and [BGI'08]), our assumptions seem much weaker. Our key exchange has a similar gap between the computation of the honest party and the adversary as prior work, while being non-interactive, implying fine-grained PKC.
	

\end{abstract}

%		In 2016 Degwekar, Vaikuntanathan, and Vasudevan [DVV'16] explored a circuit-complexity-based approach, bounding computation power of adversaries by NC$^0$ or AC$^0$ for both symmetric-key and public-key primitives. While in 2017

	%The last few years have seen a resurgence in interest in fine-grained cryptography, which had its origins in fine-grained key exchanges from the 1970s (see [Merkle'78] and [BGI'08]). 
%There are two main motivations. The first is that all known c
 % and were able to prove that getting their approach to work might be hard.
 
 	%This motivates constructing cryptographic primitives from a variety of {\em unrelated} yet plausible complexity assumptions. %Ideally these assumptions would be simultaneously consistent with $P=NP$ and non-algebraic. %makes sense - if one assumption fails, another might prevail. 
 %It is thus worthwhile to build primitives from combinatorial assumptions in \emph{polynomial time} that are markedly different from the typical exponential algebraic hardness hypotheses in cryptography. %A second important goal 
 %is the search for practically efficient protocols. %Striving for security against all polynomial time adversaries often carries with it the cost of higher (though polynomial) runtime for the honest parties. Practically speaking, it makes sense to weaken the security requirement: have very efficient, hopefully linear time, honest parties, and a large but fixed polynomial cost for the adversary.

%If we make an even more believable conjecture that Zero $k$-Clique, or even the related $k$-SUM problem, require super-linear time to solve on average, we build fine-grained one-way functions and fine-grained hard-core bits.
	%
%Our contribution is to start building fine-grained one-way functions and key exchanges from fine-grained average-case problems, like $k$-sum and zero $k$-clique, as well as explore the implications. First, we characterize which problems with which properties can be used to build these one-way functions, and we show that relatively weak assumptions about fine-grained problems can yield these one-way functions; these assumptions are only that it takes super-linear time for an adversary to solve such a problem. We also show that there exist fine-grained versions of hardcore bits. 
%
%Second, we provide a novel construction for a fine-grained key exchange based on a fine-grained assumption. Where previous works had to assume random oracles or the existence of strong one-way functions to get a key-exchange computable in $n$ time secure against $O(n^2)$ adversaries, such as Merkle and Biham et al respectively, we make a much weaker assumption: zero $k$-clique is requires $\tilde \Omega(n^k)$ time to solve on average. This key exchange, like Merkle and Biham's constructions, will be non-interactive, and therefore will imply fine-grained public key encryption.
% \end{abstractpage}

\cleardoublepage

\section*{Acknowledgments}
%TODO
This is the acknowledgements section.  You should replace this with your
own acknowledgements. TODO.

I would first like to thank my advisors Vinod Vaikuntanathan and Virginia Vassilesvka Williams. A special thank you to Vinod for supporting me during my entire time at MIT, ...
%TODO

Tal Moran
%TODO

Elette Boyle
%TODO

Andrea Lincoln
%TODO

Other coauthors
%TODO

Dan Boneh
%TODO

Other members of the theory community.
%TODO

Other members of the crypto community.
%TODO

Family and Avery <3
%TODO

%%%%%%%%%%%%%%%%%%%%%%%%%%%%%%%%%%%%%%%%%%%%%%%%%%%%%%%%%%%%%%%%%%%%%%
% -*-latex-*-
