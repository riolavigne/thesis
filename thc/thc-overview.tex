% !TEX root = ../main.tex

Secure communication over an insecure network is one of the fundamental goals of cryptography. 
A fundamental solution to this problem is secure multiparty computation. Here, one commonly assumes that all parties have pairwise communication channels.
In contrast, for many real-world scenarios, the communication network is not complete, and parties can only communicate with a subset of other parties. A natural question is whether a set of parties can successfully perform a joint computation over an incomplete communication network while revealing no information about the network topology.

The problem of \emph{topology-hiding computation} (THC) 
was introduced by Moran et al. \cite{TCC:MorOrlRic15}, who showed that 
THC is possible in the setting with passive corruptions and graphs with logarithmic diameter. Further solutions improve the 
communication efficiency \cite{C:HMTZ16}, and allowed for larger classes of graphs \cite{EC:AkaMor17,ALM17}.
A natural next step is to extend these results to settings with  more powerful adversaries. Unfortunately, even a protocol in the setting with fail-corruptions (in addition to passive corruptions) must leak topological information about the graph \cite{TCC:MorOrlRic15}.

A comparison of previous works in topology-hiding communication is found in Tables~\ref{tbl:overview1}.

\begin{table}[ht]
	\centering
	\caption{Adversarial model and security assumptions of existing topology-hiding broadcast protocols. The table also shows the class of graphs for which the protocols have polynomial communication complexity in the security parameter and the number of parties.}
	\renewcommand{\arraystretch}{1.25}\setlength{\tabcolsep}{.5em}
	\begin{tabular}{ | c | c | c | c | c | }
		\hline
		Adversary & Graph & Hardness Asm. & Model &  Reference \\ \hline\hline
		
		\multirow{5}{*}{semi-honest}
		& log diam.& Trapdoor Perm. & Standard & \cite{TCC:MorOrlRic15}  \\ \cline{2-5}
		& log diam. & DDH & Standard & \cite{C:HMTZ16}\\ \cline{2-5}
		& \makecell{cycles, trees, \\ log circum.} & DDH & Standard & \cite{EC:AkaMor17} \\ \cline{2-5}
		& arbitrary & DDH or QR & Standard & \makecell{\cite{C:AkaLaVMor17} and \\ \cite{EPRINT:AkaLaVMor17}} \\ \hline
		
		fail-stop 	& arbitrary & OWF  & Trusted Hardware & \cite{BBMM18}\\ \hline
		
		\makecell{semi-malicious \\ \& fail-stop} & arbitrary & \makecell{DDH or QR \\ or LWE\footnote{LWE is possible due to a result presented in this thesis. See Section \ref{sec:lwe-pkcr}.} } & Standard & \cite{LLMMMT18} \\ \hline
		
%		\multirow{4}{*}{ {\vspace*{1.5cm}\makecell{Probabilistic \\ Unknown \\ Delays}} }
%		& \makecell{cycles, trees, \\ log circum.} & \makecell{DDH or QR \\ or LWE} & Standard & \cite{LLMMMT20} \\ \cline{2-5}
%		& arbitrary & OWF & Trusted Hardware & \cite{LLMMMT20} \\
%		\hline
	\end{tabular}
	\label{tbl:overview1}
\end{table}
%----

\subsection{Results and Techniques}
This thesis presents two results in the field of \THC. The first is a description of how to get \PKCR, a central tool used in most of the standard-model results, from the Learning-With-Errors assumption. The second discusses asynchronous settings for \THC, presenting a key impossibility result. This result implies that any of the usual adversarial models for asynchronous communication will make \THC~impossible.

\paragraph{Privately Key-Commutative Randomizeable Encryption (\PKCR).}
All of the standard-model results (i.e. based on standard cryptographic assumptions) use a special form of encryption called OR-Homomorphic \PKCR. This kind introduced in \cite{EC:AkaMor17} and used the Decisional Diffie-Hellman (DDH) assumption. Later in \cite{EPRINT:AkaLaVMor17}, it was shown you could make \PKCR~with the QR assumption. Here, we will show that OR-Homomorphic \PKCR~is also possible under the Learning-With-Errors (LWE).

\PKCR~is detailed in Section \ref{sec:pkcr-desc}.

%------ asynch ------%
\paragraph{Asynchronous settings.} All these prior results consider the \emph{fully synchronous} model, where a protocol proceeds in rounds. This model makes
two assumptions: first, the parties have access to synchronized clocks, and 
second, every message is guaranteed to be delivered within one round.
While the first assumption is reasonable in practice, as nowadays computers 
usually stay synchronized with milliseconds of variation, the second assumption 
makes protocols inherently impractical because the running time of a protocol is always counted in the number of rounds, and the round length must be chosen based on the most pessimistic bound on the message delivery time.

A first attempt would be to develop a protocol for the fully asynchronous model, where the adversary has complete control over delays. Unfortunately, we can show that \emph{any} setting where the adversary can control delays in an unbounded way, leaks information (see Section \ref{sec:impossible}).
