% !TEX root = ../main.tex
In this section, we will go over the preliminaries for \THC. This will include a primer on previous protocols and the tools necessary for them. 

\subsection{Graphs and Random Walks}
In an undirected graph $G = (V,E)$ we denote by $\nbh{\party_i}$ the neighborhood of $\party_i \in V$. The $k$-neighborhood of a party $\party_i \in V$ is the set of all parties in $V$ within distance $k$ to $\party_i$.

The following lemma from \cite{C:AkaLaVMor17} states that in an undirected connected graph $G$, the probability that a random walk of length $8|V|^3 \tau$ covers $G$ is at least $1- \frac{1}{2^\tau}$. 

\begin{lemma}[\cite{C:AkaLaVMor17}]\label{lem:randomWalkCover}
	Let $G = (V,E)$ be an undirected connected graph. Further let $\mathcal{W}(u,\tau)$ be a random variable whose value is the set of nodes covered by a random walk starting from $u$ and taking $8|V|^3 \tau$  steps. We have
	\begin{equation*}
	\Pr_\mathcal{W}[\mathcal{W}(u,\tau) = V] \ge 1- \frac{1}{2^\tau}.
	\end{equation*} 
\end{lemma}

\subsection{Random Walk Protocol from \texorpdfstring{\cite{C:AkaLaVMor17}}{[ALM17a]}}\label{sec:RandomWalks}

The most efficient protocol for THC is from \cite{C:AkaLaVMor17}, and techniques (e.g. random walks) are the only known techniques for achieving \THC~in a passive adversarial setting in polynomial communication and rounds under standard assumptions. We present a summary of it here so that it is easy to understand why \PKCR~is important and how current methods break down in an asynchronous setting.

We recall that the random walk protocol achieves security against static passive corruptions. To achieve broadcast, the protocol actually computes an OR.  Every party has an input bit: the sender inputs the broadcast bit and all other parties use 0 as input bit. Computing the OR of all those bits is thus equivalent to broadcasting the sender's message.

First, we will explain a simplified version of the protocol that is unfortunately not sound, but this gets the principal across. Each node will take its bit, encrypt it under a public key and forward it to a random neighbor. The neighbor OR's its own bit, adds a fresh public key layer, and it randomly chooses the next step in the walk that the message takes, choosing a random neighbor to forward the bit. Eventually, after about $O(\secparam n^3)$ steps, the random walk of every message will visit every node in the graph, and therefore, every message will contain the OR of all bits in the network. Now we start the backwards phase, reversing the walk and peeling off layers of encryption.

This scheme is not sound because seeing where the random walks are coming from reveals information about the graph! So, we need to disguise that information. We will do so by using correlated random walks, and will have a walk running down each direction of each edge at each step (the number of walks is then 2$\times$ number of edges). The walks are correlated, but still random. This way, at each step, each node just sees encrypted messages all under new and different keys from each of its neighbors. So, intuitively, there is no way for a node to tell anything about where a walk came from.

In more detail, and to demonstrate security, consider a single node $v$ with $d$ neighbors. During the forward phase at step $t$, $v$ gets $d$ incoming messages --- one from each of its neighbors
homomorphically OR's its bit to each, computes $d$ fresh public keys, adding a layer to each, and finally computes a random permutation $\pi_t$ on its neighbors, forwarding the message it got from neighbor $i$ to neighbor $\pi_t(i)$ and so on. During the backwards phase, node $v$ removes the public key layers it added during the corresponding forward round, and then reverses the permutation, sending the message it got from neighbor $j$ to neighbor $\pi_t^{-1}(j)$. Because all messages are encrypted under semantically secure encryption, $v$ cannot tell whether it has received a 0 or 1 from any of its neighbors, and because all of its neighbors are layering their own fresh public keys onto the messages, there is no way for $v$ to tell where that message came from or if it had seen it before. Intuitively, this gives us soundness (see \cite{C:AkaLaVMor17} for details).

Now, this protocol is also correct: every walk will, with all but negligible probability, visit every node in the network, and therefore every message will, with all but negligible probability, contain an encryption of the OR of all bits in the graph by the end of the forward phase. The backward phase then takes that message at the end of the walk, and reverses the walk exactly, popping off the public key layer that was added at each step. By the end of the backward phase, the node that started the walk gets the decryption of the message: the OR of all bits in the graph. Because all of the walks succeed, and every node started a walk, every node gets the correct output bit as desired.


\subsection{OR-Homomorphic PKCR Encryption Scheme}\label{sec:pkcr-desc}
In \cite{C:AkaLaVMor17} and \cite{EC:AkaMor17}, protocols require a public key encryption scheme with additional properties, called \emph{Privately Key Commutative and Rerandomizable encryption}.
We assume that the message space is bits. Then, a PKCR encryption scheme should be: (1) privately key commutative and (2) homomorphic with respect to the OR operation. We formally define these properties below.
\footnote{PKCR encryption was introduced in \cite{EC:AkaMor17,C:AkaLaVMor17}, where it had three additional properties: key commutativity, homomorphism and rerandomization, hence, it was called Privately Key Commutative and \emph{Rerandomizable} encryption. However, rerandomization is actually implied by the strengthened notion of homomorphism. Therefore, we decided to not include the property, but keep the name.}

Let $\PK$, $\SK$ and $\cC$ denote the public key, secret key and ciphertext spaces. As any public key encryption scheme, a PKCR scheme contains the algorithms $\keygen:\{0,1\}^* \to \PK \times \SK$, $\enc:\{0,1\} \times \PK \to \cC$ and $\dec:\cC \times \SK \to \{0,1\}$ for key generation, encryption and decryption respectively (where $\keygen$ takes as input the security parameter).

For a public-key $\pk$ and a message $m$, we denote the encryption of $m$ under $\pk$ by $\encrypted{m}{\pk}$. Furthermore, for $k$ messages $m_1,\dots,m_k$, we denote by $\encrypted{m_1,\dots,m_k}{\pk}$ a vector, containing the $k$ encryptions of messages $m_i$ under the same key $\pk$.

For an algorithm $\algorithm{A}(\cdot)$, we write $\algorithm{A}(\cdot \ ;U^*)$ whenever the randomness used in $\algorithm{A}(\cdot)$ should be made explicit and comes from a uniform distribution. By $\approx_c$ we denote that two distribution ensembles are computationally indistinguishable.

\subsubsection{Privately Key-Commutative}
We require $\PK$ to form a commutative group under the operation $\keymul$. So, given any $\pk_1, \pk_2 \in \PK$, we can efficiently compute $\pk_3 = \pk_1 \keymul \pk_2 \in \PK$ and for every $\pk$, there exists an inverse denoted $\pk^{-1}$. 
This $\pk^{-1}$ must be efficiently computable given the secret key corresponding to $\pk$.

This group must interact well with ciphertexts; there exists a pair of efficiently computable algorithms $\AddLayer : \cC \times \SK \to \cC$ and $\DelLayer : \cC \times \SK \to \cC$ such that
\begin{itemize}
	\item For every public key pair $\pk_1, \pk_2 \in \PK$ with corresponding secret keys $\sk_1$ and $\sk_2$, message $m \in \cM$, and ciphertext $c = [m]_{\pk_1}$,
	\[\AddLayer(c, \sk_2) = [m]_{\pk_1 \keymul \pk_2}.\]
	\item For every public key pair $\pk_1, \pk_2 \in \PK$ with corresponding secret keys $\sk_1$ and $\sk_2$, message $m \in \cM$, and ciphertext $c = [m]_{\pk_1}$,
	\[\DelLayer(c, \sk_2) = [m]_{\pk_1 \keymul \pk_2^{-1}}.\]
\end{itemize}
Notice that we need the secret key to perform these operations, hence the property is called \emph{privately} key-commutative.

\subsubsection{OR-Homomorphic}
We also require the encryption scheme to be OR-ho\-mo\-mor\-phic, but in such a way that parties cannot tell how many 1's or 0's were OR'd (or who OR'd them). 
We need an efficiently-evaluatable homomorphic-OR algorithm, $\homOr : \cC \times \cC \to \cC$, to satisfy the following: for every two messages $m, m' \in \{0,1\}$ and every two ciphertexts $c, c' \in \cC$ such that $\dec(c, \sk) = m$ and $\dec(c, \sk) = m'$,
\begin{gather*}
\left\{ (m,m', c,c', \pk, \enc(m \lor m', \pk; U^*)) \right\} \\
\approx_c \\
\left\{ (m,m', c,c', \pk, \homOr(c, c', \pk; U^*)) \right\}\\
\end{gather*}
Note that this is a stronger definition for homomorphism than usual; usually we only require correctness, not computational indistinguishability.

In \cite{C:HMTZ16}, \cite{EC:AkaMor17} and \cite{C:AkaLaVMor17}, the authors discuss how to get this kind of homomorphic OR under the DDH assumption, and later \cite{EPRINT:AkaLaVMor17} show how to get it with the QR assumption. For more details on other kinds of homomorphic cryptosystems that can be compiled into OR-homomorphic cryptosystems, see \cite{EPRINT:AkaLaVMor17}.

In this thesis we show how to instantiate a PKCR encryption scheme under the LWE assumption (see Section \ref{sec:lwe-pkcr}).