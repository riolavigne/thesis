%\section{Key Exchange Proofs}\label{sec:keyExchangeProofs}
%Here we put the technical proofs of our key exchange. While the intuition for these proofs is straightforward, getting the details and constants correct requires more careful attention.

\subsection{Proof of Correctness}\label{sec:proof-of-correctness}
First, we will prove Lemma \ref{lem:keyxc-is-correct}. We have restated the lemma below.
\begin{lemma}
	After running construction \ref{const:fg-interactive-keyxc}, Alice and Bob agree on a key $k$ with probability at least $1 - \frac{1}{10,000 \ell e}$.
\end{lemma}
\begin{proof}
	Since we are allowing interaction, the only way Alice and Bob can fail is if one of Alice's $\Generate(n,0)$ contains a solution that overlaps with $S_B$, one of Bob's $\Generate(n,0)$ contains a solution that overlaps with $S_A$, \emph{and} $S_A \cap S_B = \emptyset$.
	
	First, let's compute $p_0 = \Pr[S_A \cap S_B = \emptyset]$. We have $p_0 = \prod_{i = 0}^{\sqrt \ell} \left(\frac{\ell - \sqrt \ell - i}{\ell} \right)$, the chance that every time Bob chooses an element for $S_B$, he does not choose an element in $S_A$. Rearranging this expression, we have
	\[ p_0 = \prod_{i = 0}^{\sqrt \ell-1} \left(\frac{\ell - \sqrt \ell - i}{\ell} \right) = \prod_{i=0}^{\sqrt \ell-1} (1 - \frac{\sqrt{\ell} + i}{\ell})
	\le \prod_{i=0}^{\sqrt{\ell} - 1} \left( 1 - \frac{1}{\sqrt{\ell}} \right)
	= \left( 1 - \frac{1}{\sqrt{\ell}} \right)^{\sqrt{\ell}} \approx \frac{1}{e}
	\]
	
	Now, assuming that $S_A$ and $S_B$ do not intersect, we need to compute the probability that \emph{both} Alice and Bob see an incorrectly generated instance (generated by $\Generate(n,0)$, but contains a solution). Let $\epsilon_{plant} \le \frac{1}{100 \ell}$ be the planting error. Since there is no overlap between $S_A$ and $S_B$, these probabilities are independent. The probability that $S_A$ overlaps is at most $ \sqrt \ell \epsilon_{plant} \le \frac{1}{100 \sqrt \ell}$ via a union bound over all $\sqrt \ell$ instances corresponding to the indices in $S_A$. Therefore, the probability that this happens for both Alice and Bob is at most $ \frac{1}{10,000\ell} = \left(\frac{\sqrt \ell }{10,000\ell}\right)^2$.
	
	Thus, the probability that this event occurs is at most $\frac{1}{10,000 \ell e}$, and it is the only way the protocol ends without Alice and Bob agreeing on a key.
	
	Therefore, the probability Alice and Bob agree on a key at the end of the protocol is $1 - \frac{1}{10,000 \ell e}$. \qed
\end{proof}

\subsection{Proof of Soundness}\label{sec:proof-of-soundness}

Here we will go over the full proofs that an adversary, Eve, must take more time that Alice and Bob to obtain the exchanged key. First, we upper bound the time Alice and Bob take. Then, we lower-bound the time Eve requires to break the key exchange assuming that we have a $\ell(n)$-\keyER~ problem $P$.

%The following Lemma is related to the birthday paradox. In the sense that we are measuring 

\begin{lemma}
	If a problem $P$ is $\ell(n)$-\keyER~ with plant time $G(n)$, solve time $S(n)$ and lower bound $T(n)$ when $\ell(n)>100$,
	then Alice and Bob take expected time $O(\ell G(n) + \sqrt{\ell} S(n))$ to run the key exchange based on $P$.
\end{lemma}
\begin{proof}
	First, we will compute a bound on the number of times Alice and Bob need to repeat the key exchange before they match on exactly one index. Alice and Bob repeat any time there isn't exactly one overlap between $S_A$ and $S_B$ or the key exchange fails, as described in the proof of Lemma \ref{lem:keyxc-is-correct}. Since the probability of the bad event happening is small, $\le 1/(10,000 e \ell)$, we will ignore it. Instead, saying
	\begin{align*}
	\Pr[&\mbox{Key Exchange Stops after this round}]\\
	&= \Pr[\mbox{bad event}] + \Pr[\mbox{Exactly one overlap} | \mbox{ no bad event}] \cdot\Pr[\mbox{no bad event}]\\
	&\ge \frac 1 2 \Pr[\mbox{Exactly one overlap} | \mbox{ no bad event}] = \Pr[\mbox{ Exactly one overlap}]/2.
	\end{align*}
	
	\paragraph{Computing the probability that there is exactly one overlap.} Let $p_0$ and $p_1$ be the probability that there are zero overlaps and exactly 1 overlap respectively. First, using similar techniques as in the proof of Lemma \ref{lem:keyxc-is-correct}, we show that $p_0 \ge \frac 1 {e^2}$
	\[ 
	 p_0 %= \prod_{i = 0}^{\sqrt \ell-1} \left(\frac{\ell - \sqrt \ell - i}{\ell} \right)
	 = \prod_{i=0}^{\sqrt \ell-1} (1 - \frac{\sqrt{\ell} + i}{\ell})
	\ge \prod_{i=0}^{\sqrt{\ell} - 1} \left( 1 - \frac{2\sqrt{\ell}}{\ell} \right)
	= \left( 1 - \frac{2}{\sqrt{\ell}} \right)^{\sqrt{\ell}} \approx \frac{1}{e^2}.
	\]
	A combinatorial argument also tells us that $p_0 = \binom{\ell - \sqrt{\ell}}{\sqrt{\ell}} / \binom{\ell}{\sqrt{\ell}}$ since there are $\binom{\ell}{\sqrt{\ell}}$ possible ways to choose $S_A$ independent of $S_B$, but if we want to ensure no overlap between $S_A$ and $S_B$, we need to avoid the $\sqrt{\ell}$ locations in $S_B$, hence $\binom{\ell - \sqrt{\ell}}{\sqrt{\ell}}$ choices for $S_A$. Then, we have $p_1 = \sqrt{\ell} \cdot \binom{\ell - \sqrt{\ell}}{\sqrt{\ell} - 1} / \binom{\ell}{\sqrt{\ell}}$ because there are $\sqrt{\ell}$ places to choose from to overlap $S_A$ with $S_B$, and then we must avoid the $\sqrt{\ell} - 1$ locations in $S_B$ for the rest of the $\sqrt{\ell}$ elements in $S_A$.
	
	Now we will compute a bound on $p_1$ by first showing $\frac{p_1}{p_0} \ge 1$:
	\begin{align*}
	\frac{p_1}{p_0} &= \frac{\sqrt \ell \binom{\ell - \sqrt \ell}{\sqrt \ell - 1}}{\binom \ell {\sqrt \ell}} \cdot \frac{\binom \ell {\sqrt \ell}}{ \binom {\ell - \sqrt \ell}{\sqrt \ell} }\\
	&= \frac{\sqrt \ell (\ell - \sqrt \ell)!}{(\sqrt \ell - 1)!(\ell - 2\sqrt \ell + 1)!} \cdot \frac{(\sqrt \ell)!(\ell - 2\sqrt \ell)!}{(\ell - \sqrt \ell)!}\\
	&= \frac{(\sqrt \ell)^2}{\ell - 2 \sqrt \ell + 1} = \frac{\ell}{\ell - 2\sqrt \ell + 1} \ge 1
	\end{align*}
	Now, we have that $p_1 = \frac{p_1}{p_0} \cdot p_0 \ge 1 \cdot \frac{1}{e^2} \ge 1 / 10$.
	
	Finally, putting this all together, the probability that Alice and Bob stop after a round of the protocol is at least $\frac 1 {20}$. And so, we expect Alice and Bob to stop after a constant number of rounds. Each round consists of calling $\Generate$ $\ell$ times and solving $\sqrt{\ell}$ instances; so, each round takes $\ell G(n) + \sqrt{\ell}S(n)$ time. Therefore, Alice and Bob take $O(\ell G(n) + \sqrt{\ell}S(n))$.
	\qed
\end{proof}

Now, proving a lower bound on Eve's time.
\begin{lemma}
	If a problem $P$ is $\ell(n)$-\keyER~ with plant time $G(n)$, solve time $S(n)$ and lower bound $T(n)$ when $\ell(n)\ge 2^{14}$,
	then an eavesdropper Eve, when given the transcript $\vec I_T$, requires $\~\Omega(\ell(n) T(n))$ to solve for the shared key.
\end{lemma}
\begin{proof}
	This proof requires two steps: first, if Eve can figure out the shared key in time $\PFT{\ell(n)T(n)}$ time with advantage $\delta_{Eve}$, then she can also figure out the index in $\PFT{\ell(n)T(n)}$ time with probability $\delta_{Eve}/4$. Then, if Eve can compute the index with advantage $\delta_{Eve}/4$, we can use Eve to solve the list-version of $P$ in $\PFT{\ell(n)T(n)}$ with probability $\delta_{Eve}/16$, which is a contradiction to the list-hardness of our problem.
	
	\paragraph{Finding a bit finds the index.} This is just the Goldreich-Levin (GL) trick used in classical cryptography to convert OWFs to OWFs with a hardcore bit. We have to be careful in this scenario since the security reduction for GL requires polynomial overhead ($O(N^2)$). However, this is only because we are trying to find $N$ bits based off of linear combinations of those bits. If instead we were trying to find $\poly \log N$ bits, we would only require $\poly\log N$ time to do so with this trick. $i \in \ell(n)$ is an index, so $|i| = \log(\ell(n))$. Because $\ell(n)$ is polynomial in $n$, $|i|$ is polynomial in the $\log$ of $n$, therefore, using the same techniques as used in the proof of Theorem \ref{thm:fine-grained-GL}, being able to determine $i \xor r$ with $\delta$ advantage allows us to determine $i$ in the same amount of time, with probability $\delta/4$.
	
	\paragraph{Finding the index solves $P$} Now, let $\vec I = ( I_1, \ldots, I_\ell )$ be an instance of the list problem for $P$: for a random index $i$, $I_i \gets D_1$, and for all other $j \neq i, I_j \gets D_0$.
	Because $P$ is generalized splittable, we can take every $I_i$ and turn it into a list of $m$ instances. With probability $1 - \ell \epsilon_{split}$, we turn $\vec I$ to $m$ different instances: for every $c \in [m]$, $\vec I^{(c)} = ( (I_1^{(1,c)}, I_1^{(2,c)}), \ldots  (I_\ell^{(1,c)}, I_\ell^{(2, c)} ))$.
	For all $c$ and $j \neq i$, $(I_j^{(1, c)}, I_j^{(2,c)}) \sim D_0 \x D_0$, and for at least one $c^* \in [m]$, $( I_i^{(1, c^*)}, I_i^{(2,c^*)} ) \sim D_1 \x D_1$. Because $P$ is plantable, for $\sqrt \ell - 1$ random coordinates $h \in [\ell]$, for all $c \in [m]$, we will change $I_h^{(1, c)}$ to $I_h'^{(1, c)} \sim \Generate(n,1)$, and for $\sqrt \ell - 1$ random coordinates $g \in [\ell]$, disjoint from all $h$'s, for every $c \in [m]$, we will similarly plant solutions in the second list, changing $I_g^{(2, c)}$ to $I_g'^{(2, c)} \sim \Generate(n,1)$.
	
	Note that there are $\ell$ instances and Eve returns a single index. We can verify the correctness by brute forcing a single instance in the list instance. When $\ell$ is polynomial in $n$ then the time to brute force is polynomially smaller than the time required to solve the list instance. We will need to brute force $m$ of these instances (one for each of the $m$ produced pairs of lists). When $\ell/m = n^{-\Omega(1)}$, the total time for all the brute forces is polynomially smaller than the time required for solving a single list instance. This is how we deal with the ``dummy'' instances produced with by the splittable construction. 
	
	Now, notice that we have changed the list version of the problem into $m$ different lists of pairs of instances, $\left\{ \left(\vec I_1^{(c)}, \vec I_2^{(c)}\right) \right\}_{c \in [m]}$, and there exists a $c^*$ such that the $c^*$'th list is distributed, with probability $O(1 - 1/\sqrt \ell)$, indistinguishably to the transcript of a successful key exchange between Alice and Bob. We planted $\sqrt \ell - 1$ solutions into random indices, and as long as we avoided the index with the solution (which happens with probability $1 - \frac 2 {\sqrt \ell}$), the rest of the pairs will be of the form $D_0 \x D_0$ with exactly one coordinate of overlapping instances with solutions. That coordinate will be the same as the index in the list problem with the solution.
	
	So, since we are assuming Eve can run in $\PFT{\ell(n) T(n)}$ time and we can create instances that look like key-exchange transcripts from list-problems, we can run Eve on each of these $m$ different list-pair problems, and as long as she answers correctly for the $c^*$ instance, we can solve our original problem in time $O((\ell(n)T(n))^{1 - \delta})$ for $\delta > 0$. This is a contradiction to the hardness of the list problem, meaning Eve's time is bounded by $\Omega(\ell(n) T(n))$.
	
	Analyzing the error in this case, when the key exchange succeeds, the total variation distance between an instance of the list problem being split and the original key-exchange transcript is bounded above by the following two sides:
	\begin{itemize}
		\item For the $c^*$ that splits the $D_1$ instance of the list into one sampled from $D_1 \x D_1$, this succeeds with probability $1 - \epsilon_{split} \cdot \ell$.
		\item Given that we successfully split, the distance between the generated pairs of lists \emph{after} we plant $\sqrt{\ell} - 1$ instances with a solution between this and the idealized list of $(D_b, D_{b'})$ instances with one $(D_1, D_1)$, $\sqrt{\ell} - 1$ of the form $(D_1, D_0)$ and $\sqrt{\ell}-1$ of the form $(D_0, D_1)$ is at most $\frac{2}{\sqrt \ell} + (1 - \frac 2 {\sqrt \ell})(\sqrt \ell \cdot \epsilon_{plant}) \le \frac{2}{\sqrt \ell} + \sqrt{\ell}\epsilon_{plant}$.
		\item For the generated instances generated in a successful key exchange transcript, the error between this and the idealized list-pairs (described above) is at most $\ell \cdot \epsilon_{plant}$.
		\item Recall that $\epsilon_{plant}, \epsilon_{split} \le \frac{1}{100 \ell}$ and that $\ell \ge 2^{14}$. So, combined, the key-exchange transcript distribution and splitting the list-hard problem distribution are indistinguishable with probability at most
		\begin{align*}
		1 &- (\epsilon_{split} \ell  + \frac{2}{\sqrt \ell} + \sqrt{\ell} \epsilon_{plant} + \ell \epsilon_{plant})\\
		&= 1 - (\ell(\epsilon_{split} + \epsilon_{plant}) + \sqrt{\ell} \epsilon_{plant} + \frac{2}{\sqrt \ell})\\
		&\ge 1 - (\ell(\frac 2 {128 \ell}) + \frac{1}{128 \sqrt \ell} + \frac 2 {\sqrt \ell})\\
		&\ge 1 - (\frac{2}{128} + \frac{2}{128} + \frac{1}{128^2}) = 1 - \frac{1}{32} - \frac{1}{2^{14}}\\
		&> 1 - \frac{1}{31}
		\end{align*}
	\end{itemize}
	Therefore, the total variation distance between key-exchange transcripts and the transformed ACLH instances is at most $\frac 1 {31}$.
	
	Now, recall that if we have a $\PFT{\ell(n)T(n)}$ algorithm $E$ that resolves the single-bit key with advantage $\delta$, then there exists a $\PFT{\ell(n) T(n)}$ algorithm $E^*$ that resolves the index of the key exchange transcript with probability $\delta/4$. Let $Transf$ be the algorithm that transforms an ACLH instance $\vec I$ to the key-exchange transcript (with TVD from a successful key-exchange transcript of $\frac 1 {34}$) Therefore, the probability that we fool Eve into solving our ACLH problem is
	\[ \Pr[ E^*(Transf(\vec I)) = i ] \ge \delta/4 - \frac 1{31} \ge \frac{1}{16} - \frac 1 {31} > \frac 1 {34} \]
	Now, since the ACLH problem $P$ allows for $\PFT{\ell(n)T(n)}$ adversaries to have advantage at most $\frac 1 {34}$, this is a contradiction. Therefore, there does not exist a $\PFT{\ell(n)T(n)}$ eavesdropping adversary that can resolve the single bit key with advantage $\frac 1 4$ (so resolving the key with probability $1/2 + 1/4 = 3/4$).
	\qed
\end{proof}

We note that the range, $R \approx n^{6k}$ in the above corollary may be considered to be ``too large'' if you believe the hardness in the problem comes from a range where were are expected to get one solution with probability 1/2 ($R = O(n^k)$). So, in the next corollary, we address that problem, getting the key exchange using this much smaller range.

We will now provide the proof for Corollary \ref{cor:kcliqueKeyExchange}. We will repeat The text of the Corollary here. 

\textbf{Corollary \ref{cor:kcliqueKeyExchange}.}
\emph{	Given the \strongzkc~ over range $R = \ell(n)^2 n^{2k}$, there exists a\\ $\fgkeyxc{\ell(n)T(n)}{1/4}{\insig(n)}$, where Alice and Bob can exchange a sub-polynomial-sized key in time $\tilde{O}\left(n^{k}\sqrt{\ell(n)} + n^2\ell(n)\right)$ for every polynomial $\ell(n)= n ^{\Omega(1)}$.}
	
\emph{	There also exists a $\ell(n)T(n)$-fine-grained public-key cryptosystem, where we can encrypt a sub-polynomial sized message in time $\tilde{O}\left(n^{k}\sqrt{\ell(n)} + n^2\ell(n)\right)$.}
	
\emph{	Both of these protocols are optimized when $\ell(n) = n^{2k-4}$.}
	
\begin{proof}
	This comes from the fact that \strongzkc implies that \zkclique~is a KER problem by Theorem \ref{thm:zkcsplittable}, Theorem \ref{thm:zkcAvgListHard}, and Theorem \ref{thm:zkcPlantable}.  So we can use construction \ref{const:fg-interactive-keyxc} to get the key-exchange by Theorem \ref{thm:ATTimpPKE} and Claim \ref{clm:medium-strong-owf}. %We can also use it to get the fine-grained public-key cryptosystem by theorem \ref{thm:fg-pkc}.
	
	The optimization comes from minimizing $\tilde{O}\left(n^{k}\sqrt{\ell(n)} + n^2\ell(n)\right)$, which is simply to set $n^k \sqrt{\ell(n)} = n^2 \ell(n)$. This results in $\ell(n) = n^{2k - 4}$.
	
	The gap between honest parties and dishonest parties is computed as follows. Honest parties take $H(n) = \~O(\ell(n)n^2) = \~O(n^{2k - 2})$. Dishonest parties take $E(n) = \~O(\ell(n)n^k) = \~O(n^{3k - 4})$. We have that $E(n) = H(n)^t$ where $t = \frac{3k-4}{2k-2}$, which approaches $1.5$ as $k \to \infty$. So, we have close to a 1.5 gap between honest parties and dishonest ones as long as we assume $T(n) = n^{k}$.
	\qed
\end{proof}

\begin{corollary}
	Given the \strongzkc~over range $R = n^{k}$, where $\ell(n)$ is polynomial, there exists a $\fgkeyxc{\ell(n)T(n)}{1/4}{\insig(n)}$, where Alice and Bob can exchange a sub-polynomial-sized key in time $\tilde{O}\left(n^{k}\sqrt{\ell(n)} + n^2\ell(n)\right)$ for every polynomial $\ell(n)= n ^{\Omega(1)}$.
	
	There also exists a $\ell(n)T(n)$-fine-grained public-key cryptosystem, where we can encrypt a sub-polynomial sized message in time $\tilde{O}\left(n^{k}\sqrt{\ell(n)} + n^2\ell(n)\right)$.
	
	Both of these protocols are optimized when $\ell(n) = n^{2k-4}$.
	\label{cor:kcliqueKeyExchangeRange}
\end{corollary}
\begin{proof}
	Using Corollary \ref{cor:kcliqueKeyExchange} and Theorem \ref{thm:rangeExtendThm} we can use the hardness of \strongzkc~ over range $R = n^{k}$ to show hardness for \strongzkc~ over range $R = \ell(n)^2n^{2k}$.
	\qed
\end{proof}