\documentclass[11pt]{article}
\pagestyle{plain}

\usepackage{fullpage}
\usepackage{amssymb,amsmath,url}%ifplatform}
\usepackage{amsthm}
\usepackage[pdfstartview=FitH,colorlinks,linkcolor=blue,citecolor=blue]{hyperref}
\usepackage{graphicx}
\usepackage{tikz}
\usepackage{pgfplots}
\usepackage{stmaryrd}
\usepackage{MnSymbol}
\usepackage[linesnumbered,ruled]{algorithm2e}
\usepackage{palatino}

\usepackage{xcolor}

\usepackage{macros}
\usepackage{float}
\floatstyle{boxed} \restylefloat{figure}

\newtheorem{construction}[theorem]{Construction}

% Complex \xxx for making notes of things to do.  Use \xxx{...} for general
% notes, and \xxx[who]{...} if you want to blame someone in particular.
% Puts text in brackets and in bold font, and normally adds a marginpar
% with the text ``xxx'' so that it is easy to find.  On the other hand, if
% the comment is in a minipage, figure, or caption, the xxx goes in the text,
% because marginpars are not possible in these situations.
{\makeatletter
	\gdef\xxxmark{%
		\expandafter\ifx\csname @mpargs\endcsname\relax % in minipage?
		\expandafter\ifx\csname @captype\endcsname\relax % in figure/caption?
		\marginpar{xxx}% not in a caption or minipage, can use marginpar
		\else
		xxx % notice trailing space
		\fi
		\else
		xxx % notice trailing space
		\fi}
	\gdef\xxx{\@ifnextchar[\xxx@lab\xxx@nolab}
	\long\gdef\xxx@lab[#1]#2{\textbf{[\xxxmark #2 ---{\sc #1}]}}
	\long\gdef\xxx@nolab#1{\textbf{[\xxxmark #1]}}
	% This turns them off:
	% \long\gdef\xxx@lab[#1]#2{}\long\gdef\xxx@nolab#1{}%
}



\newcommand{\ttildeo}{\tilde{o}}
\newcommand{\tO}{\tilde{O}}
\newcommand{\tTheta}{\tilde{\Theta}}
\newcommand{\tOmg}{\tilde{\Omega}}
\newcommand{\strongzkc}{strong zero k-clique hypothesis}
\newcommand{\Strongzkc}{strong zero k-clique hypothesis}

\def\keygen{\mathsf{KeyGen}}
\def\hash{\mathsf{Hash}}
\def\compute{\mathsf{Eval}}

\def\secp{\lambda}

\def\vecx{\mathbf{x}}
\def\vecy{\mathbf{y}}
\def\matA{\mathbf{A}}
\def\CLOSE{\emph{\textbf{CLOSE}}}
\def\FAR{\emph{\textbf{FAR}}}

\def\SETINT{\mathsf{L2}}

%%%%%%%%%%%%  Defining theorem-like environments %%%%%%%%%%
%\newtheorem{theorem}{Theorem}
%\newtheorem{corollary}{Corollary}
%\newtheorem{claim}{Claim}
%\newtheorem{conjecture}{Conjecture}

%%%%%%%%%%%% macros
\newcommand{\wmin}{w_{\mathrm{min}}}
\newcommand{\wmax}{w_{\mathrm{max}}}
\newcommand{\Ls}{\mathcal L}
\title{Fine Grained Public Key Cryptography- At some point}
\author{Rio LaVigne and Andrea Lincoln}

\begin{document}
\maketitle

\section{Introduction}

\xxx{Story: fine grained crypto needs all this stuff. }

Fine-grained complexity has helped explain the hardness of a wide variety of problems. 

The idea of fine-grained one way functions were introduced by Ball et.al. \cite{avgCaseFineGrained}. 

Fine-grained cryptography ideally would have all of the following objects listed bellow. Those in bold we have created. Those italicized have been solved previously. 

\begin{itemize}
	\item Plausibly average case hardness assumptions on which to base cryptographic constructions. 
	\begin{itemize}
		\item \textit{Worst case to average case reductions to show hardness for problems.} \cite{avgCaseFineGrained}
		\item Showing problems hard from satisfiability. 
		\item \textbf{Using plausible average case hypotheses. }
		\item Assumptions that have a fine-grained implications for quantum computers as well. 
	\end{itemize}
	\item  ``Symmetric'' Fine-grained Cryptography 
		\begin{itemize}
		\item \textbf{Fine-grained one way functions (FGOWFs)}
		\item \textbf{FGOWFs with hardcore bits}
		\item Fine-grained pseudo-random generators (FGPRGs)
		\end{itemize}
	\item ``Asymmetric'' Fine-grained Cryptography 
		\begin{itemize}
		\item Fine-grained trap-door functions
		\item \textbf{Fine-grained key transfer}
		\item Fine-grained public key cryptography
		\item Fine-grained identity based encryption
		\end{itemize}
	\item Advanced constructions
		\begin{itemize}
		\item Oblivious transfer
		\item Homomorphic encryption
		\end{itemize}
\end{itemize}


% Decrease the space between bibliography items.
\let\realbibitem=\bibitem
\def\bibitem{\par \vspace{-1.2ex}\realbibitem}

\bibliographystyle{alpha}
\bibliography{mybib} 
\end{document}