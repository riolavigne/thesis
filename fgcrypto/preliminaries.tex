%\section{Preliminaries: Model of Computation and Definitions}\label{sec:prelims}

The running times of all algorithms are analyzed in the word-RAM model of computation, where simple operations such as $+, -, \cdot, \,$ bit-shifting, and memory access all require a single time-step.

Just as in normal exponential-gap cryptography we have a notion of probabilistic polynomial-time (PPT) adversaries, we can similarly define an adversary that runs in time less than expected for our fine-grained polynomial-time solvable problems. This notion is something we call probabilistic fine-grained time (or $\PFT{}$). Using this notion makes it easier to define things like OWFs and doesn't require carrying around time parameters through every reduction.

\begin{definition}
	An algorithm $\cA$ is an $T(n)$ probabilistic fine-grained time, $\PFT{T(n)}$, algorithm if there exists a constant $\delta > 0$ such that $\cA$ runs in time $O(T(n)^{1-\delta})$.
\end{definition}
Note that in this definition, assuming $T(n) = \Omega(n)$, any sub-polynomial factors can be absorbed into $\delta$.

Additionally, we will want a notion of \emph{negligibility} that cryptography has. Recall that a function $\negl(n)$ is negligible if for all polynomials $Q(n)$ and sufficiently large $n$, $\negl(n) < 1/Q(n)$. We will have a similar notion here, but we will use the words \emph{significant} and \emph{insignificant} corresponding to non-negligible and negligible respectively.


\begin{definition}
	A function $\sig(n)$ is \emph{significant} if 
	\[\sig(n) \ge \frac{1}{p\left(n\right)}\]
	for all polynomials $p$. A function $\insig(n)$ is \emph{insignificant} if for all significant functions $\sig(n)$ and sufficiently large $n$,
	\[\insig(n) < \sig(n).\]
\end{definition}

Note that for every polynomial $f$, $1/f(n)$ is insignificant. Also notice that if a probability is significant for an event to occur after some process, then we only need to run that process a sub-polynomial number of times before the event will happen almost certainly. This means our run-time doesn't increase even in a fine-grained sense; i.e. we can boost the probability of success of a randomized algorithm running in $\~O(T(n))$ from $1/\log(n)$ to $O(1)$ just by repeating it $O(\log(n))$ times, and still run in $\~O(T(n))$ time (note that ` $\tilde\ $ ' suppresses all sub-polynomial factors in this work).

\subsection{Fine-Grained Symmetric Crypto Primitives}

Ball et all defined fine-grained one-way functions (OWFs) in their work from 2017 \cite{avgCaseFineGrained}. They parameterize their OWFs with two functions: an inversion-time function $T(n)$ (how long it takes to \emph{invert} the function on $n$ bits), and an probability-of-inversion function $\epsilon$; given $T(n)^{1-\delta'}$ time, the probability any adversary can invert is $\epsilon(T(n)^{1-\delta'})$. The computation time is implicitly defined to be anything noticeably less than the time to invert: there exists a $\delta > 0$ and algorithm running in time $T(n)^{1-\delta}$ such that the algorithm can evaluate $f$.

%\begin{definition}[$(\alpha, \epsilon)$-one-way functions]
%	A function $f:\{0,1\}^* \to \{0,1\}^*$ is \emph{$(t, \epsilon)$-one-way} if, for some $\delta > 0$, it can be evaluated on $n$ bits in $O(T(n))$ time, but for any $\delta' > 0$ and any adversary $\cA$ running in $O(T(n)^{1 - \delta'})$ time and all sufficiently large $n$,
%	\[\Pr_{x \gets \{0,1\}^n}\left[\cA(f(x)) \in f\inv(f(x))\right] \le \epsilon(n, \delta').\]
%\end{definition}

\begin{definition}[$(\delta, \epsilon)$-one-way functions]
	A function $f:\{0,1\}^* \to \{0,1\}^*$ is \emph{$(\delta, \epsilon)$-one-way} if, for some $\delta > 0$, it can be evaluated on $n$ bits in $O(T(n)^{1 - \delta})$ time, but for any $\delta' > 0$ and for any adversary $\cA$ running in $O(T(n)^{1 - \delta'})$ time and all sufficiently large $n$,
	\[\Pr_{x \gets \{0,1\}^n}\left[\cA(f(x)) \in f\inv(f(x))\right] \le \epsilon(n, \delta).\]
\end{definition}

Using our notation of $\PFT{T(n)}$, we will similarly define OWFs, but with one fewer parameter. We will only be caring about $T(n)$, the time to invert, and assume that the probability an adversary running in time less than $T(n)$ inverts with less time is insignificant. We will show later, in section \ref{sec:fg-owfs}, that we can compile fine-grained one-way functions with probability of inversion $\epsilon \le 1-\frac{1}{n^{o(1)}}$ into ones with insignificant probability of inversion. So, it makes sense to drop this parameter in most cases.

\begin{definition}\label{def:fg-owf}
	A function $f:\{0,1\}^* \to \{0,1\}^*$ is $T(n)$ fine-grained one-way (is an $T(n)$-FGOWF) if there exists a constant $\delta > 0$ such that it takes time $T(n)^{1-\delta}$ to evaluate $f$ on any input, and there exists a function $\eps(n) \in \insig(n)$, and for all $\PFT{T(n)}$ adversaries $\cA$,
	\[ \Pr_{x \gets \{0,1\}^n}\left[\cA(f(x)) \in f\inv(f(x))\right] \le \eps(n). \]
\end{definition}

With traditional notions of cryptography there was always an exponential or at least super-polynomial gap between the amount of time required to evaluate and invert one-way functions. In the fine-grained setting we have a polynomial \emph{gap} to consider.
\begin{definition}\label{def:gap}
	The (relative) \emph{gap} of an $T(n)$ fine-grained one-way function $f$ is the constant $\delta > 0$ such that it takes $T(n)^{1-\delta}$ to compute $f$ but for all $\PFT{T(n)}$ adversaries $\cA$,
	\[ \Pr_{x \gets \{0,1\}^n}\left[\cA(f(x)) \in f\inv(f(x))\right] \le \insig(n). \]
\end{definition}

\subsection{Fine-Grained Asymmetric Crypto Primitives}
In this paper, we will propose a fine-grained key exchange. First, we will show how to do it in an interactive manner, and then remove the interaction. Removing this interaction means that it implies fine-grained public key encryption! Here we will define both of these notions: a fine-grained non-interactive key exchange, and a fine-grained, CPA-secure public-key cryptosystem.

First, consider the definition of a key exchange, with interaction. This definition is modified from \cite{BGI08} to match our notation.
We will be referring to a transcript generated by Alice and Bob and the randomness they used to generate it as a ``random transcript''. 

\begin{definition}[Fine-Grained Key Exchange]\label{def:fgkeyxc}
	A $\fgkeyxc{T(n)}{\alpha}{\gamma}$ is a protocol, $\Pi$, between two parties $A$ and $B$ such that the following properties hold
	\begin{itemize}
		\item Correctness. At the end of the protocol, $A$ and $B$ output the same bit ($b_A = b_B$) except with probability $\gamma$;
		\[ \Pr_{\Pi, A, B}[b_A = b_B] \ge 1 - \gamma \]
		This probability is taken over the randomness of the protocol, $A$, and $B$.
		
		\item Efficiency. There exists a constant $\delta > 0$ such that the protocol for both parties takes time $\~O(T(n)^{1 - \delta})$.
		
		\item Security. Over the randomness of $\Pi$, $A$, and $B$, we have that for all $\PFT{T(n)}$ eavesdroppers $E$ has advantage $\alpha$ of guessing the shared key after seeing a random transcript.  Where a transcript of the protocol $\Pi$ is denoted $\Pi(A,B)$.
		\[ \Pr_{A,B}[E(\Pi(A,B)) = b_B] \le \frac 1 2 + \alpha \]
	\end{itemize}
	A $\strongfgkeyxc{T(n)}$ is a $\fgkeyxc{T(n)}{\alpha}{\gamma}$ where $\alpha$ and $\gamma$ are insignificant. The key exchange is considered \emph{weak} if it is not strong.
\end{definition}

%We will show that via simple boosting techniques, as long as $1 - \gamma > \frac 1 2 + \alpha$ (by at least a constant), we can get a $\fgkeyxc{T(n)}{\insig(n)}{\insig(n)}$.

%\xxx{Rio: I think we might want to remove this definition in favor of just stating that we can remove interaction to get a FG PK Cryptosystem.}

%Our definition of non-interactive key exchange formalizes the (lack of) communication that our parties make and makes it easier to convert it into a full-blown public key system. Notice that the definition of security in this sense is the same as before, and although sometimes the exchange will fail (produce different keys for each party), we can bootstrap using the same boosting techniques to increase the correctness in the interactive version. We can also get multi-bit keys through repetition.
%
%\begin{definition}
%	A non-interactive $\fgkeyxc{T(n)}{\alpha}{\gamma}$ consists of three algorithms: $\setup$, $\keygen$, and $\sharedKey$.
%	\begin{description}
%		\item [$\setup(1^n)$] Outputs a set of public parameters $\mpk$.
%		\item [$\keygen(ID, \mpk)$] Outputs a public-secret key pair $(pk, sk)$.
%		\item [$\sharedKey(ID_1, pk_1, ID_2, sk_2)$] Outputs the shared key $k$.
%
%	\end{description}
%	It must have the following correctness, efficiency, and security properties:
%	\begin{itemize}
%		\item Correctness. So,
%		\[ \Pr_{\keygen, \sharedKey}[\sharedKey(ID_1, pk_1, ID_2, sk_2) = \sharedKey(ID_2, pk_2, ID_1, sk_1)] \ge 1 - \gamma \]
%		
%		\item Efficiency. There exists a constant $\delta > 0$ such that all algorithms, $\setup$, $\keygen$ and $\sharedKey$ take time at most $\~O(T(n)^{1 - \delta})$.
%		\item Security. Over the randomness of $\setup$, $\keygen$, and $\sharedKey$, we have that for all $\PFT{T(n)}$ eavesdroppers $E$ has advantage $\alpha$ of guessing the shared key after seeing two random public keys:
%		\[ \Pr_{\setup, \keygen, \sharedKey}[E(pk_1, pk_2) = \sharedKey(pk_1, sk_2) \mid (pk_1, sk_1), (pk_1, sk_2) \gets \keygen(\mpk)] \le \frac 1 2 + \alpha \]
%	\end{itemize}
%\end{definition}

%Note that non-interactive key exchange implies public key cryptography.  
This particular security guarantee protects against chosen plaintext attacks. But first, we need to define what we mean by a fine-grained public key cryptosystem.

\begin{definition}\label{def:fg-pkc}
	An \emph{$T(n)$-fine-grained public-key cryptosystem} has the following three algorithms.
	\begin{description}
		\item $\keygen(1^\secp)$ Outputs a public-secret key pair $(pk,sk)$.
		\item $\enc(pk, m)$ Outputs an encryption of $m$, $c$.
		\item $\dec(sk, c)$ Outputs a decryption of $c$, $m$.
	\end{description}
	These algorithms must have the following properties:
	\begin{itemize}
		\item They are efficient. There exists a constant $\delta > 0$ such that all three algorithms run in time $O\left(T(n)^{1-\delta}\right)$.
		\item They are correct. For all messages $m$,
		\begin{align*}
		\Pr_{\keygen,\enc,\dec}&[\dec(sk, \enc(pk,m)) = m \vert (pk,sk) \gets \keygen(1^\lambda)] \ge\\
		&1 - \insig(n). 
		\end{align*}
%		\[ \Pr_{\keygen,\enc,\dec}[\dec(sk, \enc(pk,m)) = m \vert (pk,sk) \gets \keygen(1^\lambda)] \ge 1 - \insig(n). \]
	\end{itemize}

The cryptosystem is \emph{CPA-secure} if any $\PFT{T(n)}$ adversary $\cA$ has an insignificant advantage in winning the following game:
	\begin{enumerate}
		\item Setup. A challenger $\cC$ runs $\keygen(1^n)$ to get a pair of keys, $(pk, sk)$, and sends $pk$ to $\cA$.
		\item Challenge. $\cA$ gives two messages $m_0$ and $m_1$ to the challenger. The challenger chooses a random bit $b \getsr \{0,1\}$ and returns $c \gets \enc(pk, m_b)$ to $\cA$.
		\item Guess. $\cA$ outputs a guess $b'$ and wins if $b' = b$.
	\end{enumerate}
\end{definition}
