% !TEX root = ../main.tex

%\section{Fine-Grained Key Exchange}
%\label{sec:FineGrainedKeyExchange}

Now we will explain a construction for a \emph{key exchange} using general distributions. We will then specify the properties we need for problems to generate a secure key exchange. We will finally generate a key exchange using the strong \zkclique~hypothesis.

\newcommand{\keyER}{KER}

Before doing this, we will define a class of problems as being Key Exchange Ready (KER).

\begin{definition}[Key Exchange Ready (KER)]
	A problem $P$ is $\ell(n)$-\keyER~ with generate time $G(n)$, solve time $S(n)$ and lower bound solving time $T(n)$ if
	\begin{itemize}
		\item there is an algorithm which runs in $\tilde\Theta(S(n)))$ time that determines if an instance of $P$ of size $n$ has a solution or not,
		\item the problem is $(\ell(n), \delta_{LH})$-\ACLH~where $\delta_{LH} \le \frac 1 {34}$,
		\item is Generalized Splittable with error $\leq 1/(128 \ell(n))$ to the problem $P'$ and,
		\item $P'$ is plantable in time $G(n)$ with error $\leq 1/(128 \ell(n))$.
		\item $\ell(n)T(n) \in \~ \omega\left(\ell(n)G(n) + \sqrt{\ell(n)}S(n)\right)$, and
		\item there exists an $n'$ such that for all $n \ge n'$, $\ell(n) \ge 2^{14}$.
	\end{itemize}
\end{definition}

\subsection{Description of a Weak Fine-Grained Interactive Key Exchange}

The high level description of the key exchange is as follows. Alice and Bob each produce $\ell(n) - \sqrt{\ell(n)}$ instances using $\Generate(n,0)$ and $\sqrt{\ell(n)}$ generate instances with $\Generate(n,1)$. Alice then shuffles the list of $\ell(n)$ instances so that those with solutions are randomly distributed. Bob does the same thing (with his own private randomness). Call the set of indices that Alice chooses to plant solutions $S_A$ and the set Bob picks $S_B$. The likely size of $S_A \cap S_B$ is $1$. The index $S_A \cap S_B$ is the basis for the key.

Alice determines the index $S_A \cap S_B$ by brute forcing all problems at indices $S_A$ that Bob published. Bob can brute force all problems at indices $S_B$ that Alice published and learn the set  $S_A \cap S_B$. 

If after brute forcing for instances either Alice or Bob find a number of solutions not equal to 1 then they communicate this and repeat the procedure (using interaction). They only need to repeat a constant number of times. 

More formally our key exchange does the following:
\begin{construction}[Weak Fine-Grained Interactive Key Exchange]\label{const:fg-interactive-keyxc}
	A fine-grained key exchange for exhanging a single bit key.
	\begin{itemize}
		\item $\setup(1^n)$: output $\mpk= (n, \ell(n))$ and $\ell(n)>2^{14}$.
		\item $\keygen(\mpk)$: Alice and Bob both get parameters $(n,\ell)$.
		\begin{itemize}
			\item Alice generates a random $S_A \subset [\ell]$, $|S_A| = \sqrt \ell$. She generates a list of instances $\vec I_A = (I_A^1, \ldots, I_A^\ell)$ where for all $i \in S_A$, $I_i = \Generate(n,1)$ and for all $i \nin S_A$, $I_A^i = \Generate(n,0)$ (using Alice's private randomness). Alice publishes $\vec I_A$ and a random vector $\vec v \getsr \{0,1\}^{ \log \ell }$.
			\item Bob computes $\vec I_B = (I_B^1, \ldots, I_B^\ell)$ similarly: generating a random $S_B \subset [\ell]$ of size $\sqrt \ell$ and for every instance $I_j \in \vec I_B$, if $j \in S_B$, $I_j = \Generate(n,1)$ and if $j \nin S_B$, $I_j = \Generate(n,0)$. Bob publishes $\vec I_B$.
		\end{itemize}
		\item Compute shared key: Alice receives $\vec I_B$ and Bob receives $\vec I_A$.
		\begin{itemize}
			\item Alice computes what she believes is $S_A \cap S_B$: for every $i \in S_A$, she brute force checks if $I_B^i$ has a solution or not. For each $i$ that does, she records in list $L_A$. 
			\item Bob computes what he thinks to be $S_B \cap S_A$: for every $j \in S_B$, he checks if $I_A^j$ has a solution. For each that does, he records it in $L_B$.
		\end{itemize}
		\item Check: Alice takes her private list $L_A$: if $|L_A| \neq 1$, Alice publishes that the exchange failed. Bob does the same thing with his list $L_B$: if $|L_B| \neq 1$, Bob publishes that the exchange failed. If either Alice or Bob gave or recieved a failure, they both know, and go back to the $\keygen$ step.
		
		If no failure occurred, then $|L_A| = |L_B| = 1$. Alice interprets the index $i \in L_A$ as a vector and computes $i \cdot \vec v$ as her key. Bob uses the index in $j \in L_B$ and also computes $j \cdot \vec v$. With high probability, $i = j$ and so the keys are the same.
	\end{itemize}
\end{construction}


\subsection{Correctness and Soundness of the Key Exchange}
We want to show that with high probability, once the key exchange succeeds, both Alice and Bob get the same shared index.

\begin{lemma}\label{lem:keyxc-is-correct}
	After running construction \ref{const:fg-interactive-keyxc}, Alice and Bob agree on a key $k$ with probability at least $1 - \frac{1}{10,000 \ell e}$.
\end{lemma}
\begin{proof}
	Since we are allowing interaction, the only way Alice and Bob can fail is if one of Alice's $\Generate(n,0)$ contains a solution that overlaps with $S_B$, one of Bob's $\Generate(n,0)$ contains a solution that overlaps with $S_A$, \emph{and} $S_A \cap S_B = \emptyset$.
	
	First, let's compute $p_0 = \Pr[S_A \cap S_B = \emptyset]$. We have $p_0 = \prod_{i = 0}^{\sqrt \ell} \left(\frac{\ell - \sqrt \ell - i}{\ell} \right)$, the chance that every time Bob chooses an element for $S_B$, he does not choose an element in $S_A$. Rearranging this expression, we have
	\[ p_0 = \prod_{i = 0}^{\sqrt \ell-1} \left(\frac{\ell - \sqrt \ell - i}{\ell} \right) = \prod_{i=0}^{\sqrt \ell-1} (1 - \frac{\sqrt{\ell} + i}{\ell})
	\le \prod_{i=0}^{\sqrt{\ell} - 1} \left( 1 - \frac{1}{\sqrt{\ell}} \right)
	= \left( 1 - \frac{1}{\sqrt{\ell}} \right)^{\sqrt{\ell}} \approx \frac{1}{e}
	\]
	
	Now, assuming that $S_A$ and $S_B$ do not intersect, we need to compute the probability that \emph{both} Alice and Bob see an incorrectly generated instance (generated by $\Generate(n,0)$, but contains a solution). Let $\epsilon_{plant} \le \frac{1}{100 \ell}$ be the planting error. Since there is no overlap between $S_A$ and $S_B$, these probabilities are independent. The probability that $S_A$ overlaps is at most $ \sqrt \ell \epsilon_{plant} \le \frac{1}{100 \sqrt \ell}$ via a union bound over all $\sqrt \ell$ instances corresponding to the indices in $S_A$. Therefore, the probability that this happens for both Alice and Bob is at most $ \frac{1}{10,000\ell} = \left(\frac{\sqrt \ell }{10,000\ell}\right)^2$.
	
	Thus, the probability that this event occurs is at most $\frac{1}{10,000 \ell e}$, and it is the only way the protocol ends without Alice and Bob agreeing on a key.
	
	Therefore, the probability Alice and Bob agree on a key at the end of the protocol is $1 - \frac{1}{10,000 \ell e}$. \qed
\end{proof}
%\begin{proof-sketch}
%	We notice that the only way Alice and Bob fail to exchange a key is if they \emph{both} generate a solution accidentally in each other's sets (that is Alice generates exactly one accidental solution in $S_B$ and Bob in $S_A$), and $S_A \cap S_B = \emptyset$. All other `failures' are detectable in this interactive case and simply require Alice and Bob to run the protocol again. So, we just bound the probability this happens, and since $\epsilon_{plant} \le \frac{1}{100 \sqrt \ell}$, we get the bound $1 - \frac{1}{10,000 \ell e}$. The full proof can be found in Appendix \ref{sec:proof-of-correctness}.
%\end{proof-sketch}

%\subsection{Proof of Soundness}
We next show that the key-exchange results in gaps in running time and success probability between Alice and Bob and Eve. Then, we will show that this scheme can be boosted in a fine-grained way to get larger probability gaps (a higher chance that Bob and Alice exchange a key and lower chance Eve gets it) while preserving the running time gaps. 

First, we need to show that the time Alice and Bob take to compute a shared key is less (in a fine-grained sense) than the time it takes Eve, given the public transcript, to figure out the shared key. This includes the number of times we expect Alice and Bob to need to repeat the process before getting a usable key.

\paragraph{Time for Alice and Bob.}

\begin{lemma}\label{lem:alice-bob-time}
	If a problem $P$ is $\ell(n)$-\keyER~ with plant time $G(n)$, solve time $S(n)$ and lower bound $T(n)$ when $\ell(n)>100$,
	then Alice and Bob take expected time $O(\ell G(n) + \sqrt{\ell} S(n))$ to run the key exchange.
\end{lemma}
\begin{proof}
	First, we will compute a bound on the number of times Alice and Bob need to repeat the key exchange before they match on exactly one index. Alice and Bob repeat any time there isn't exactly one overlap between $S_A$ and $S_B$ or the key exchange fails, as described in the proof of Lemma \ref{lem:keyxc-is-correct}. Since the probability of the bad event happening is small, $\le 1/(10,000 e \ell)$, we will ignore it. Instead, saying
	\begin{align*}
	\Pr[&\mbox{Key Exchange Stops after this round}]\\
	&= \Pr[\mbox{bad event}] + \Pr[\mbox{Exactly one overlap} | \mbox{ no bad event}] \cdot\Pr[\mbox{no bad event}]\\
	&\ge \frac 1 2 \Pr[\mbox{Exactly one overlap} | \mbox{ no bad event}] = \Pr[\mbox{ Exactly one overlap}]/2.
	\end{align*}
	
	\paragraph{Computing the probability that there is exactly one overlap.} Let $p_0$ and $p_1$ be the probability that there are zero overlaps and exactly 1 overlap respectively. First, using similar techniques as in the proof of Lemma \ref{lem:keyxc-is-correct}, we show that $p_0 \ge \frac 1 {e^2}$
	\[ 
	p_0 %= \prod_{i = 0}^{\sqrt \ell-1} \left(\frac{\ell - \sqrt \ell - i}{\ell} \right)
	= \prod_{i=0}^{\sqrt \ell-1} (1 - \frac{\sqrt{\ell} + i}{\ell})
	\ge \prod_{i=0}^{\sqrt{\ell} - 1} \left( 1 - \frac{2\sqrt{\ell}}{\ell} \right)
	= \left( 1 - \frac{2}{\sqrt{\ell}} \right)^{\sqrt{\ell}} \approx \frac{1}{e^2}.
	\]
	A combinatorial argument also tells us that $p_0 = \binom{\ell - \sqrt{\ell}}{\sqrt{\ell}} / \binom{\ell}{\sqrt{\ell}}$ since there are $\binom{\ell}{\sqrt{\ell}}$ possible ways to choose $S_A$ independent of $S_B$, but if we want to ensure no overlap between $S_A$ and $S_B$, we need to avoid the $\sqrt{\ell}$ locations in $S_B$, hence $\binom{\ell - \sqrt{\ell}}{\sqrt{\ell}}$ choices for $S_A$. Then, we have $p_1 = \sqrt{\ell} \cdot \binom{\ell - \sqrt{\ell}}{\sqrt{\ell} - 1} / \binom{\ell}{\sqrt{\ell}}$ because there are $\sqrt{\ell}$ places to choose from to overlap $S_A$ with $S_B$, and then we must avoid the $\sqrt{\ell} - 1$ locations in $S_B$ for the rest of the $\sqrt{\ell}$ elements in $S_A$.
	
	Now we will compute a bound on $p_1$ by first showing $\frac{p_1}{p_0} \ge 1$:
	\begin{align*}
	\frac{p_1}{p_0} &= \frac{\sqrt \ell \binom{\ell - \sqrt \ell}{\sqrt \ell - 1}}{\binom \ell {\sqrt \ell}} \cdot \frac{\binom \ell {\sqrt \ell}}{ \binom {\ell - \sqrt \ell}{\sqrt \ell} }\\
	&= \frac{\sqrt \ell (\ell - \sqrt \ell)!}{(\sqrt \ell - 1)!(\ell - 2\sqrt \ell + 1)!} \cdot \frac{(\sqrt \ell)!(\ell - 2\sqrt \ell)!}{(\ell - \sqrt \ell)!}\\
	&= \frac{(\sqrt \ell)^2}{\ell - 2 \sqrt \ell + 1} = \frac{\ell}{\ell - 2\sqrt \ell + 1} \ge 1
	\end{align*}
	Now, we have that $p_1 = \frac{p_1}{p_0} \cdot p_0 \ge 1 \cdot \frac{1}{e^2} \ge 1 / 10$.
	
	Finally, putting this all together, the probability that Alice and Bob stop after a round of the protocol is at least $\frac 1 {20}$. And so, we expect Alice and Bob to stop after a constant number of rounds. Each round consists of calling $\Generate$ $\ell$ times and solving $\sqrt{\ell}$ instances; so, each round takes $\ell G(n) + \sqrt{\ell}S(n)$ time. Therefore, Alice and Bob take $O(\ell G(n) + \sqrt{\ell}S(n))$.
	\qed
\end{proof}
%\begin{proof-sketch}
%	It is easy to see that one iteration of the key exchange protocol requires $\ell G(n)$ time to generate the $\ell$ problems, and then $\sqrt{\ell}S(n)$ time to brute-force solve $\sqrt{\ell}$ instances of $P$. However, we need to prove that we only iterate this key-exchange a constant number of times. This part is a simple application of the birthday paradox, showing that we expect $S_A$ and $S_B$ to intersect in exactly one place with constant probability, and then applying the accuracy of our planting functionality (which succeeds with probability $1 - \epsilon_{plant}$).
%	The full proof can be found in Appendix \ref{sec:proof-of-soundness}.
%\end{proof-sketch}

\paragraph{Time for Eve.}

\begin{lemma}\label{lem:eve-time}
	If a problem $P$ is $\ell(n)$-\keyER~ with plant time $G(n)$, solve time $S(n)$ and lower bound $T(n)$ when $\ell(n)\ge 2^{14}$,
	then an eavesdropper Eve, when given the transcript $\vec I_T$, requires $\~\Omega(\ell(n) T(n))$ time to solve for the shared key with probability $\frac{1}{2}+\sig(n)$.
\end{lemma}
\begin{proof}
	This proof requires two steps: first, if Eve can figure out the shared key in time $\PFT{\ell(n)T(n)}$ time with advantage $\delta_{Eve}$, then she can also figure out the index in $\PFT{\ell(n)T(n)}$ time with probability $\delta_{Eve}/4$. Then, if Eve can compute the index with advantage $\delta_{Eve}/4$, we can use Eve to solve the list-version of $P$ in $\PFT{\ell(n)T(n)}$ with probability $\delta_{Eve}/16$, which is a contradiction to the list-hardness of our problem.
	
	\paragraph{Finding a bit finds the index.} This is just the Goldreich-Levin (GL) trick used in classical cryptography to convert OWFs to OWFs with a hardcore bit. We have to be careful in this scenario since the security reduction for GL requires polynomial overhead ($O(N^2)$). However, this is only because we are trying to find $N$ bits based off of linear combinations of those bits. If instead we were trying to find $\poly \log N$ bits, we would only require $\poly\log N$ time to do so with this trick. $i \in \ell(n)$ is an index, so $|i| = \log(\ell(n))$. Because $\ell(n)$ is polynomial in $n$, $|i|$ is polynomial in the $\log$ of $n$, therefore, using the same techniques as used in the proof of Theorem \ref{thm:fine-grained-GL}, being able to determine $i \xor r$ with $\delta$ advantage allows us to determine $i$ in the same amount of time, with probability $\delta/4$.
	
	\paragraph{Finding the index solves $P$} Now, let $\vec I = ( I_1, \ldots, I_\ell )$ be an instance of the list problem for $P$: for a random index $i$, $I_i \gets D_1$, and for all other $j \neq i, I_j \gets D_0$.
	Because $P$ is generalized splittable, we can take every $I_i$ and turn it into a list of $m$ instances. With probability $1 - \ell \epsilon_{split}$, we turn $\vec I$ to $m$ different instances: for every $c \in [m]$, $\vec I^{(c)} = ( (I_1^{(1,c)}, I_1^{(2,c)}), \ldots  (I_\ell^{(1,c)}, I_\ell^{(2, c)} ))$.
	For all $c$ and $j \neq i$, $(I_j^{(1, c)}, I_j^{(2,c)}) \sim D_0 \x D_0$, and for at least one $c^* \in [m]$, $( I_i^{(1, c^*)}, I_i^{(2,c^*)} ) \sim D_1 \x D_1$. Because $P$ is plantable, for $\sqrt \ell - 1$ random coordinates $h \in [\ell]$, for all $c \in [m]$, we will change $I_h^{(1, c)}$ to $I_h'^{(1, c)} \sim \Generate(n,1)$, and for $\sqrt \ell - 1$ random coordinates $g \in [\ell]$, disjoint from all $h$'s, for every $c \in [m]$, we will similarly plant solutions in the second list, changing $I_g^{(2, c)}$ to $I_g'^{(2, c)} \sim \Generate(n,1)$.
	
	Note that there are $\ell$ instances and Eve returns a single index. We can verify the correctness by brute forcing a single instance in the list instance. When $\ell$ is polynomial in $n$ then the time to brute force is polynomially smaller than the time required to solve the list instance. We will need to brute force $m$ of these instances (one for each of the $m$ produced pairs of lists). When $\ell/m = n^{-\Omega(1)}$, the total time for all the brute forces is polynomially smaller than the time required for solving a single list instance. This is how we deal with the ``dummy'' instances produced with by the splittable construction. 
	
	Now, notice that we have changed the list version of the problem into $m$ different lists of pairs of instances, $\left\{ \left(\vec I_1^{(c)}, \vec I_2^{(c)}\right) \right\}_{c \in [m]}$, and there exists a $c^*$ such that the $c^*$'th list is distributed, with probability $O(1 - 1/\sqrt \ell)$, indistinguishably to the transcript of a successful key exchange between Alice and Bob. We planted $\sqrt \ell - 1$ solutions into random indices, and as long as we avoided the index with the solution (which happens with probability $1 - \frac 2 {\sqrt \ell}$), the rest of the pairs will be of the form $D_0 \x D_0$ with exactly one coordinate of overlapping instances with solutions. That coordinate will be the same as the index in the list problem with the solution.
	
	So, since we are assuming Eve can run in $\PFT{\ell(n) T(n)}$ time and we can create instances that look like key-exchange transcripts from list-problems, we can run Eve on each of these $m$ different list-pair problems, and as long as she answers correctly for the $c^*$ instance, we can solve our original problem in time $O((\ell(n)T(n))^{1 - \delta})$ for $\delta > 0$. This is a contradiction to the hardness of the list problem, meaning Eve's time is bounded by $\Omega(\ell(n) T(n))$.
	
	Analyzing the error in this case, when the key exchange succeeds, the total variation distance between an instance of the list problem being split and the original key-exchange transcript is bounded above by the following two sides:
	\begin{itemize}
		\item For the $c^*$ that splits the $D_1$ instance of the list into one sampled from $D_1 \x D_1$, this succeeds with probability $1 - \epsilon_{split} \cdot \ell$.
		\item Given that we successfully split, the distance between the generated pairs of lists \emph{after} we plant $\sqrt{\ell} - 1$ instances with a solution between this and the idealized list of $(D_b, D_{b'})$ instances with one $(D_1, D_1)$, $\sqrt{\ell} - 1$ of the form $(D_1, D_0)$ and $\sqrt{\ell}-1$ of the form $(D_0, D_1)$ is at most $\frac{2}{\sqrt \ell} + (1 - \frac 2 {\sqrt \ell})(\sqrt \ell \cdot \epsilon_{plant}) \le \frac{2}{\sqrt \ell} + \sqrt{\ell}\epsilon_{plant}$.
		\item For the generated instances generated in a successful key exchange transcript, the error between this and the idealized list-pairs (described above) is at most $\ell \cdot \epsilon_{plant}$.
		\item Recall that $\epsilon_{plant}, \epsilon_{split} \le \frac{1}{100 \ell}$ and that $\ell \ge 2^{14}$. So, combined, the key-exchange transcript distribution and splitting the list-hard problem distribution are indistinguishable with probability at most
		\begin{align*}
		1 &- (\epsilon_{split} \ell  + \frac{2}{\sqrt \ell} + \sqrt{\ell} \epsilon_{plant} + \ell \epsilon_{plant})\\
		&= 1 - (\ell(\epsilon_{split} + \epsilon_{plant}) + \sqrt{\ell} \epsilon_{plant} + \frac{2}{\sqrt \ell})\\
		&\ge 1 - (\ell(\frac 2 {128 \ell}) + \frac{1}{128 \sqrt \ell} + \frac 2 {\sqrt \ell})\\
		&\ge 1 - (\frac{2}{128} + \frac{2}{128} + \frac{1}{128^2}) = 1 - \frac{1}{32} - \frac{1}{2^{14}}\\
		&> 1 - \frac{1}{31}
		\end{align*}
	\end{itemize}
	Therefore, the total variation distance between key-exchange transcripts and the transformed ACLH instances is at most $\frac 1 {31}$.
	
	Now, recall that if we have a $\PFT{\ell(n)T(n)}$ algorithm $E$ that resolves the single-bit key with advantage $\delta$, then there exists a $\PFT{\ell(n) T(n)}$ algorithm $E^*$ that resolves the index of the key exchange transcript with probability $\delta/4$. Let $Transf$ be the algorithm that transforms an ACLH instance $\vec I$ to the key-exchange transcript (with TVD from a successful key-exchange transcript of $\frac 1 {34}$) Therefore, the probability that we fool Eve into solving our ACLH problem is
	\[ \Pr[ E^*(Transf(\vec I)) = i ] \ge \delta/4 - \frac 1{31} \ge \frac{1}{16} - \frac 1 {31} > \frac 1 {34} \]
	Now, since the ACLH problem $P$ allows for $\PFT{\ell(n)T(n)}$ adversaries to have advantage at most $\frac 1 {34}$, this is a contradiction. Therefore, there does not exist a $\PFT{\ell(n)T(n)}$ eavesdropping adversary that can resolve the single bit key with advantage $\frac 1 4$ (so resolving the key with probability $1/2 + 1/4 = 3/4$).
\end{proof}
We note that the range, $R \approx n^{6k}$ in the above corollary may be considered to be ``too large'' if you believe the hardness in the problem comes from a range where were are expected to get one solution with probability 1/2 ($R = O(n^k)$). So, in the next corollary, we address that problem, getting the key exchange using this much smaller range.
%\begin{proof-sketch}
%	This is proved in two steps. First, if Eve can determine the shared key in time $\PFT{\ell(n)T(n)}$ with advantage $\delta_{Eve}$, then she can also figure out the index in $\PFT{\ell(n)T(n)}$ time with probability $\delta_{Eve}/4$. Second, if Eve can compute the index with advantage $\delta_{Eve}/4$, we can use Eve to solve the list-version of $P$ in $\PFT{\ell(n)T(n)}$ with probability $\delta_{Eve}/16$, which is a contradiction to the list-hardness of our problem. This first part follows from a fine-grained Goldreich-Levin hardcore-bit theorem, Theorem \ref{thm:fine-grained-GL}.
%	
%	The second part, proving that once Eve has the index, then she can solve an instance of $P$, uses the fact that $P$ is list-hard, generalized splittable, and plantable. Intuitively, since $P$ is already list hard, we will start with a list of average problem instances $(I_1, \ldots, I_{\ell})$, and our goal will be to have Eve tell us which instance (index) has a solution. We apply the splittable property to this list to get lists of pairs of problems. For one of these lists of pairs, there will exist an index where both instances have solutions. These lists of pairs will \emph{almost} look like the transcript between Alice and Bob during the key exchange: if $I$ had a solution then there should be one index such that both instances in a pair have a solution. Now, we just need to plant $\sqrt{\ell} - 1$ solutions in the left instances and $\sqrt{\ell} - 1$ on the right, and this will be indistinguishable from a transcript between Alice and Bob. If Eve can find the index of the pair with solutions, we can quickly check that she is right (because the instances inside the list are relatively small), and simply return that index.
%	
%	The full proof can be found in Appendix \ref{sec:proof-of-soundness}.
%\end{proof-sketch}

Now, we can put all of these together to get a weak fine-grained key exchange. We will then boost it to be a strong fine-grained key exchange (see the Definition \ref{def:fgkeyxc} for weak versus strong in this setting).

\begin{theorem}\label{thm:fg-pkc}
	If a problem $P$ is $\ell(n)$-\keyER~ with plant time $G(n)$, solve time $S(n)$ and lower bound $T(n)$ when $\ell(n)\ge2^{14}$,
	then construction \ref{const:fg-interactive-keyxc} is a $((\ell(n)T(n), \alpha, \gamma)$-$\mathsf{FG}\mbox{-}\mathsf{KeyExchange}$, with $\gamma \le \frac{1}{10,000 \ell(n)e}$ and $\alpha \le \frac 1 4$.
	\label{thm:ATTimpPKE}
\end{theorem}
\begin{proof}
	This is a simple combination of the correctness of the protocol, and the fact that an eavesdropper must take more time than the honest parties. We have that the $\Pr[b_A = b_B] \ge 1 - \frac{1}{10,000 \ell e}$, implying $\gamma \le \frac{1}{10,000 \ell e}$ from Lemma \ref{lem:keyxc-is-correct}. We have that Alice and Bob take time $O(\ell(n) G(n) + \sqrt{\ell(n)}S(n))$ and Eve must take time $\~\Omega(\ell(n)T(n))$ to get an advantage larger than $\frac 1 4$ by Lemmas \ref{lem:alice-bob-time} and \ref{lem:eve-time}. Because $P$ is \keyER~, $\ell(n)T(n) \in \~\omega\left(\ell(n)G(n) + \sqrt{\ell(n)}S(n)\right)$, implying there exists $\delta > 0$ so that $\ell(n)G(n) + \sqrt{\ell(n)}S(n) \in \~O(\ell(n)T(n)^{1 - \delta})$. So, we have correctness, efficiency and security.
\end{proof}

Next, we are going to amplify the security of this key exchange using parallel repetition, drawing off of strategies from \cite{DNR04} and \cite{BIN97}.
%TODO: fix the intro part

\begin{theorem}\label{thm:amplifyKeyXC}
	If a weak $\fgkeyxc{\ell(n)T(n)}{\alpha}{\gamma}$ exists where $\gamma = O\left(\frac{1}{n^c}\right)$ for some constant $c>0$, but $\alpha = O(1)$, then a $\strongfgkeyxc{\ell(n)T(n)}$ also exists.
\end{theorem}
\begin{proof}
	Using techniques from \cite{BIN97}, we will phrase breaking this key exchange as an eavesdropper as an honest-verifier two-round parallel repetition game. The original game as a $\PFT{\ell(n)T(n)}$ prover $P$ and verifier $V$. $V$ generates an honest transcript of the key exchange between Alice and Bob and sends this transcript to $P$. Note that the single-bit key sent in this protocol is uniformly distributed. $P$ wins if $P$ can output the key correctly. Now, because any eavesdropper running $\PFT{\ell(n)T(n)}$ only has advantage $\alpha$ of determining the key, the prover $P$ has probability at most $\frac 1 2 + \alpha$ of winning this game. Let $\beta = \frac 1 2 + \alpha$. By Theorem 4.1 of \cite{BIN97}, if we instead have $V$ generate $m$ parallel repetitions ($m$ independent transcripts of the key exchange), then the probability that $P$ can find \emph{all} of the keys in $\PFT{\ell(n)T(n)}$ is less than $\frac{16}{1 - \beta} \cdot e^{-m (1 - \beta^2)/256}$ (which is larger than $\frac{32}{(1 - \beta)} \cdot e^{-mc(1 - \beta^2) / 256}$).
	
	Let $m = \frac{512}{1 - \beta^2} \cdot \log(n)$, which is sub-polynomial in $n$ because $\beta$ is at least constant. and we get that the probability the prover succeeds in finding all $m$ keys is at most $\beta' = O\left( \frac{1}{n^2} \right)$. Now, we need this to work while there is some error $\gamma$. Note that $\gamma$ is at most $1/n^c$, so the probability we get an error in any of the $m$ instances of the key exchange is $\left( 1 - \gamma \right)^{d\cdot \log(n)}$ for some constant $d$. Asymptotically, this is $e^{-\gamma d \log(n)} = n^{-\gamma d}$. This is where we required $\gamma$ to be so small: because $\gamma = O(1/n^c)$, this probability of failure is $o(\sqrt{\gamma})$, which is still insignificant.
	
	Now, we need to turn this $m$ parallel-repetition back into a key exchange. We will first do that by employing our fine-grained Goldreich-Levin method: the weak key-exchange will be run $m$ times in parallel and Alice will additionally send a uniformly random $m$-length binary vector, $\vec r$. The key will be the $m$ keys, $\vec k = (k_1, \ldots, k_m)$ dot-producted with $\vec r$: $\vec k \cdot \vec r$. Because $m$ is sub-polynomial in $n$ and the Goldreich-Levin security reduction only requires $\~O(m^2)$ time, $\vec k \cdot \vec r$ is a fine-grained hard-core bit for the transcript. Therefore, an eavesdropper will have advantage at most $\frac 1 2 + \insig(n)$ in determining the shared key.
\end{proof}

\begin{remark} It is not obvious how to amplify correctness \emph{and} security of a fine-grained key exchange at the same time. If we have a weak $\fgkeyxc{\ell(n)T(n)}{\alpha}{\gamma}$, where $\alpha = \insig(n)$ but $\gamma = O(1)$, then we can use a standard repetition error-correcting code to amplify $\gamma$. That is, we can run the key exchange $\log^2(n)$ times to get $\log^2(n)$ keys (most of which will agree between Alice and Bob), and to send a message with these keys, send that message $\log^2(n)$ times. With all but negligible probability, the decrypted message will agree with the sent message a majority of the time.
Since with very high probability the adversary cannot recover any of the keys in $\PFT{\ell(n)T(n)}$ time, this repetition scheme is still secure.

As shown in Theorem \ref{thm:amplifyKeyXC}, we can also amplify a key exchange that has constant correctness and polynomial soundness to one with $1 - \insig(n)$ correctness and polynomial soundness. However, it is unclear how to amplify both at the same time in a fine-grained manner.
\end{remark}

\begin{corollary}\label{cor:strongFGKeyXc}
	If a problem $P$ is $\ell(n)$-\keyER, then a $\textsf{Strong }(\ell(n)T(n))$-$\textsf{FG}$-\textsf{Key\-Ex\-change} exists.
\end{corollary}
\begin{proof}
	The probability of error in Construction \ref{const:fg-interactive-keyxc} is at most $\frac{1}{10,000 \ell(n) e}$, and $\ell(n) = n^{\Omega(1)}$ (due to the fact that $\ell(n)$ comes from the definition of list-hard, Definition \ref{def:aclh}. The probability a $\PFT{\ell(n)T(n)}$ eavesdropper has of resolving the key is $\frac 1 2 + \alpha$ where $\alpha\leq \frac 1 4$. This means our $\beta \leq \frac 3 4 = O(1)$. Since the clauses in theorem \ref{thm:amplifyKeyXC} are met, a $\strongfgkeyxc{\ell(n)T(n)}$ exists.
\end{proof}

Finally, using the fact that Alice and Bob do not use each other's messages to produce their own in Construction \ref{const:fg-interactive-keyxc}, we prove that we can remove all interaction through repetition and get a $T(n)$-fine-grained public key cryptosystem.

\begin{lemma}\label{lem:non-interactive-const}
	Construction \ref{const:fg-interactive-keyxc} does not need interaction.
\end{lemma}
\begin{proof}
	There is a constant probability that the construction fails at each round. So, Alice and Bob will simply run the protocol $c\cdot\log(n)$ times in parallel and take the key generated from the first successful exchange. There are two errors to keep track of: the chance that Alice and Bob's $S_A$ and $S_B$ do not intersect in exactly one spot and the probability that an instance was generated with a false-positive. Since $\epsilon_{plant}$ is so small ($O(1/n^{\Omega(1)})$), we do not need to worry about the false-positives (the chance of generating one is insignificant). So, the error we are concerned with is the chance that none of the $c \log(n)$ instances of the key exchange end up having $S_A$ and $S_B$ overlapping in exactly 1 entry. This happens with probability at most $\Pr[\mbox{no overlap } c\log(n) \mbox{ times}] = \left( \Pr[\mbox{no overlap once}] \right)^{c \log n} \leq O(1/n^c)$, which is also insignificant. Therefore, the chance this key exchange fails is at most $1 - (\frac{c\log(n)}{10,000 \ell e} + \frac{1}{n^c}) = 1 - \insig(n)$.
\end{proof}

\begin{theorem}\label{thm:fg-pkc-exists}
	 If a problem $P$ is $\ell(n)$-\keyER, then a $\ell(n) \cdot T(n)$-fine-grained public key cryptosystem exists.
\end{theorem}
\begin{proof}
	First, consider an amplified, non-interactive version of Construction \ref{const:fg-interactive-keyxc} (combination of Corollary \ref{cor:strongFGKeyXc} and lemma \ref{lem:non-interactive-const}): Alice and Bob run the protocol $m$ times in parallel, where $m = \subpoly(n)$.
	The $m$-bit key they agree is the vector of keys exchanged. We now define the three algorithms for a fine-grained public key cryptosystem.
	\begin{itemize}
		\item $\keygen(1^n)$: run Bob's half of the non-interactive protocol $m$ times, generating $m$ collections of $c\log(n)$ lists of $\ell(n)$ instances of $P$: $\{(\vec I_B^{(1,i)}, \ldots,$ $\vec I_B^{(c\log(n), i)} )\}_{i \in [m]}$. Each list $\vec I_B^{(j,i)}$ has a random set $S_B^{(j,i)} \subset [\ell]$ where instances $I_k \in \vec I_B^{(j,i)}$ are from $\Generate(n,1)$ if $k \in S_B^{(j,i)}$ and from $\Generate(n,0)$ otherwise. The public key is $pk = \{(\vec I_B^{(1,i)}, \ldots, \vec I_B^{(c\log(n), i)} )\}_{i \in [m]}$ and the secret key is $sk = \{(S_B^{(1,i)}, \ldots,$ $S_B^{(c\log(n),i)})\}_{i \in m}$.
		\item $\enc(pk, \vec m \in \{0,1\}^m)$: run Alice's half of the protocol $m$ times, solving for the shared key, and then encrypting a message under that key. More formally, generate $m$ lists of $c\log(n)$ sets $S_A^{(j,i)} \subset [\ell]$ such that $|S_A^{(j,i)}| = \sqrt{\ell}$. Then, generate lists of instances $\vec I_A^{(j,i)}$ for $i \in \{1,\ldots, m\}$ and $j \in \{1, \ldots, c\log(n)\}$, where for every instance $I_{k}^{(j,i)} \in \vec I_A^{(j,i)}$, $I_{k}^{(j,i)} = \Generate(n,1)$ if $k \in S_A^{(j,i)}$ and otherwise use $\Generate(n,0)$. Then, for all of these $m \cdot c\log(n)$ instances, generate a random vector $\vec r^{i,j}$ of length $\log(\ell)$ for the final part of the key exchange. Now, for each of these $m$ instances, compute the shared key as in Construction the non-interactive version \ref{const:fg-interactive-keyxc} (see lemma \ref{lem:non-interactive-const}), to get a vector of keys $\vec k = (k_1, \ldots, k_m)$. If any of these exchanges fail, output $\bot$.
		Finally, let the ciphertext $ct = \left( \{((\vec I_A^{i,1}, \vec r^{i, 1}), \ldots, (\vec I_A^{i,c\log(n)}, \vec r^{i, c \log(n)}))\}_{i \in [m]}, \vec k \xor \vec m \right)$.
		\item $\dec(sk, ct)$: Bob computes the shared key and decrypts the message. Formally, let $ct = \left(\{((\vec I_A^{i,1}, \vec r^{i, 1}), \ldots, (\vec I_A^{i,c\log(n)}, \vec r^{i, c \log(n)}))\}_{i \in [m]}, \vec c^*  \right)$. Bob computes the shared key just like Alice to get $\vec k$, and then decrypts the message with $\vec k \xor c^* = \vec m'$. Output the $m$-bit message $\vec m'$.
	\end{itemize}

	Now we will prove this is indeed a fine-grained public key cryptosystem. First, because the key exchange took Alice and Bob $\PFT{\ell(n)T(n)}$ time, this scheme is efficient, requiring at most $(\ell(n)T(n))^{1 - \delta} \cdot m \cdot (c \log n) = \~O(\ell(n)T(n)^{(1 - \delta)})$ for constant $\delta > 0$.
	
	Next, the scheme is correct. This comes directly from the fact that the key exchange succeeds with probability $1 - \insig(n)$.
	
	Lastly, the scheme is secure. This is a simple reduction from the security game to an eavesdropper. For sake of contradiction, let Eve be a $\PFT{\ell(n)T(n)}$ adversary that can win the CPA-security game as in Definition \ref{def:fg-pkc} with probability $\frac 1 2 + \epsilon$ where $\epsilon = \sig(n)$. Eve can then directly win the key exchange with the same advantage because the message and public key the challenger gives to Eve contains a key exchange (contains multiple key exchanges, but we can simulate all but one).
\end{proof}

\begin{corollary}
	Given the \strongzkc~over the range $R =$ $\ell(n)^2 n^{2k}$, there exists a $\fgkeyxc{\ell(n)T(n)}{1/4}{\insig(n)}$, where Alice and Bob can exchange a sub-polynomial-sized key in time $\tilde{O}\left(n^{k}\sqrt{\ell(n)} + n^2\ell(n)\right)$ for every polynomial $\ell(n)= n ^{\Omega(1)}$.
	
	There also exists a $\ell(n)T(n)$-fine-grained public-key cryptosystem, where we can encrypt a sub-polynomial sized message in time $\tilde{O}\left(n^{k}\sqrt{\ell(n)} + n^2\ell(n)\right)$.
	
	Both of these protocols are optimized when $\ell(n) = n^{2k-4}$.
	\label{cor:kcliqueKeyExchange}
\end{corollary}
\begin{proof}
	This comes from the fact that \strongzkc implies that \zkclique~is a KER problem by Theorem \ref{thm:zkcsplittable}, Theorem \ref{thm:zkcAvgListHard}, and Theorem \ref{thm:zkcPlantable}.  So we can use construction \ref{const:fg-interactive-keyxc} to get the key-exchange by Theorem \ref{thm:ATTimpPKE} and Claim \ref{clm:medium-strong-owf}. %We can also use it to get the fine-grained public-key cryptosystem by theorem \ref{thm:fg-pkc}.
	
	The optimization comes from minimizing $\tilde{O}\left(n^{k}\sqrt{\ell(n)} + n^2\ell(n)\right)$, which is simply to set $n^k \sqrt{\ell(n)} = n^2 \ell(n)$. This results in $\ell(n) = n^{2k - 4}$.
	
	The gap between honest parties and dishonest parties is computed as follows. Honest parties take $H(n) = \~O(\ell(n)n^2) = \~O(n^{2k - 2})$. Dishonest parties take $E(n) = \~O(\ell(n)n^k) = \~O(n^{3k - 4})$. We have that $E(n) = H(n)^t$ where $t = \frac{3k-4}{2k-2}$, which approaches $1.5$ as $k \to \infty$. So, we have close to a 1.5 gap between honest parties and dishonest ones as long as we assume $T(n) = n^{k}$.
\end{proof}

The \zThclique~hypothesis (the Zero Triangle hypothesis) is generally better believed than the \zkclique~hypothesis for larger $k$. Note that even with the strong \zThclique~hypothesis we get a key exchange with a gap in the running times of Alice and Bob vs Eve. In this case, the gap is $t = 5/4 = 1.2$.

Also, since the \zkclique-hypothesis may be more believable over a smaller range, say $R = n^k$, we can combine Corollary \ref{cor:kcliqueKeyExchange} and Theorem \ref{thm:rangeExtendThm} to get the below result: even with a relatively small range, we can still build a fine-grained public-key cryptosystem.

\begin{corollary}
	Given the \strongzkc~over range $R = n^{k}$, there exists a $\ell(n)T(n)$-fine-grained public-key cryptosystem, where we can encrypt a sub-polynomial sized message in time $\tilde{O}\left(n^{k}\sqrt{\ell(n)} + n^2\ell(n)\right)$.
\end{corollary}