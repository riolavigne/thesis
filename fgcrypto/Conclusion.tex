\section{Open Problems}
We have begun the process of porting ideas from fine-grained complexity into fine-grained cryptography. We do this via building these constructions, improving definitions, and coming up with plausible assumptions. 

There are two primary avenues of open problems from here. Strengthening assumptions and building more crypto primatives. 

\paragraph{Strengthing Assumptions} 
Previous work of Ball et. al. shows worst case to average case reductions, however, not for problems that are \Plantable \cite{avgCaseFineGrained}. Showing a worst case to average case reduction for a problem that is \Plantable would give fine-grained one way functions. 

\xxx{TODO citations notably anyone who says average case exp time is hard}

Showing a reduction from an Average Case Exponential Time Hypothesis to a \Plantable problem would generate a fine-grained one way function from an assumption of the hardness of SAT. Showing that average case hardness for a problem that is \Plantable, Average Case Self-Reducible and Splittable would imply a fine-grained key exchange exists. Note that this avoids previous barriers of building a key exchange from an NP-hard problem by only showing a polynomial gap in the key exchange. 


\xxx{TODO citations, notably le gall}

The problems of zero k-clique and k-sum for large enough do not yet have linear time quantum algorithms. In fact the best algorithms for zero k-clique take time $n^{k/2-o(1)}$ and the best algorithms for $k$-sum take time $n^{(\lceil k/2 \rceil /2) -o(1)}$ on quantum computers. Can faster algorithms be found? If not, can fine-grained quantum hardness be argued? 

\paragraph{Building More Cryptographic Primitives} 
Can a key-exchange be built from a problem that is only \Plantable and Average-Case Self-Reducible? Can a key-exchange be built using something that is only \Plantable and Splittable? 

Can a fine-grained key exchange be built from a fine-grained assumption that has a gap larger than $N$ vs $N^{1.5}$ between the running times of Alice and Bob vs Eve? 

Can more advanced cryptographic objects like homomorphic encryption be built more simply, or more efficiently to run in exchange for only offering polynomial security?