\section{Lemmas for the key exchange}


\begin{lemma}
		If a problem $P$ is $\ell(n)$-\keyER~ with plant time $G(n)$, solve time $S(n)$ and lower bound $T(n)$ with $\ell(n)>100$ 	
	then there is a key-exchange in which Alice and Bob successfully communicate a key, $key$, of length $\log(\ell(n)\lg^3(n))$ bits in time $\tilde{O}(\sqrt{\ell(n)}S(n) + \ell(n)G(n))$ with probability $\geq 1-\lg(n)^3/(100 \sqrt{\ell(n)})$. Let $I_T$ be the full transcript that Alice and Bob write and $r$ all the random bits used to generate it. For all $\PFT{\ell(n) T(n)}$ algorithms $A$ 
	$$Pr_{r}[A(I_T)=key]<1/n^{\lg(n)}.$$
	\label{lem:keyExchangeEveGuess}
\end{lemma}
\begin{proof}
	
%\xxx{TODO: oh god past andrea that is not how this works XD. }
	
Using lemma \ref{lem:keyexchangebasic} we can send $\lg^3(n)$ keys, $key_1,\ldots, key_{\lg(n)^3}$ using that system each independently. These can all be concatenated into one long new key $key$. Eve getting each of these independent values correct has probability $\leq (9/10)^{\lg^3(n)}$ \textbf{TODO: this is wrong. The prob of eve getting all independent instances correct is ???? but almost certainly not  $\leq (9/10)^{\lg^3(n)}$. Hopefully it can be made to $1/(subpoly)$, that should be more than enough for us.}. Alice and Bob succeed with probability $\geq 1-\lg(n)^3/(100 \sqrt{\ell(n)})$ on communicating all these bits. 
\end{proof}


We now need to insist that the running time $T(n)\ell(n)$ is sufficiently large. Note that any polynomial will do. 
\begin{lemma}
		If a problem $P$ is $\ell(n)$-\keyER~ with plant time $G(n)$, solve time $S(n)$ and lower bound $T(n)$  with $\ell(n)>100$
then there is a key-exchange in which Alice and Bob successfully communicate a key, $key$, of length $1$ bit in time $\tilde{O}(\sqrt{\ell(n)}S(n) + \ell(n)G(n))$ with probability $\geq 1-\lg(n)^3/(100 \sqrt{\ell(n)})$. Let $I_T$ be the full transcript that Alice and Bob write and $r$ all the random bits used to generate it. For all $\PFT{T(n)\ell(n)}$ algorithms $A$ and for all functions $f(n)=\left(T(n)\ell(n)\right)^{o(1)}$
$$Pr_{r}[A(I_T)=key]<1/2+1/f(n).$$
	\label{lem:keyExchangeHCBit}
\end{lemma}
\begin{proof}
We will use the same construction as Goldreich-Levin. After the protocol in Lemma \ref{lem:keyExchangeEveGuess} Alice and Bob agree on the key, $key$, Alice will announce $\vec{v}\in \{0,1\}^{\log(s(n))\lg^3(n)}$ uniformly random bits. Alice and Bob will use the bit $key \cdot \vec{v}$.

Using the construction of Goldreich-Levin (from \cite{hardCoreBitsAndXorLemmaFromGL}) we can show that the probability of Eve guessing this bit must be $\leq 1-1/f(n)$ with overhead that is $\tilde{O}(f(n)lg(\ell)\lg^3(n))$, this gives us a bound on Eve's time decreased by a factor of $f(n)lg(\ell)\lg^3(n)$. So no $\PFT{T(n)\ell(n)/\left( f(n)lg(\ell)\lg^3(n) \right)}$ algorithm can achieve probability $1-1/f(n)$. If  $f(n)=\left(T(n)\ell(n)\right)^{o(1)}$ and $\lg(n) = \left(T(n)\ell(n)\right)^{o(1)}$ then $\PFT{T(n)\ell(n)/\left( f(n)lg(\ell)\lg^3(n) \right)}$ is the same as $\PFT{T(n)\ell(n)}$. Because $T(n) = \Omega(n)$ by the properties of $T(n)$-\ACIH problems we have that $\lg(n) = \left(T(n)\ell(n)\right)^{o(1)}$. 
\end{proof}


\begin{lemma}
		If a problem $P$ is $\ell(n)$-\keyER~ with plant time $G(n)$, solve time $S(n)$ and lower bound $T(n)$ with $\ell(n)>100\lg^3(n)$
then there is a key-exchange in which Alice and Bob successfully communicate a key, $key$, of length $1$ bit in time $\tilde{O}(\sqrt{\ell(n)}S(n) + \ell(n)G(n))$ with probability $\geq 1-n^{\lg(n)}$. Let $I_T$ be the full transcript that Alice and Bob write and $r$ all the random bits used to generate it. For all $\PFT{T(n)\ell(n)}$ algorithms $A$ and for all functions $f(n)=\left(T(n)\ell(n)\right)^{o(1)}$
$$Pr_{r}[A(I_T)=key]<1/2+1/f(n).$$	
%	\label{lem:keyExchangeAllErrorsLow}
\end{lemma}
\begin{proof}
We will use xor's to allow Alice and Bob to check that they actually agree. Our construction is related to the constructions from Vazirani Vazirani and Goldreich Levin \cite{xorLemmaVazirani,hardCoreBitsAndXorLemmaFromGL}. Specifically, using calls to the procedure from Lemma \ref{lem:keyExchangeHCBit}, call this procedure $P$. 

So Alice and Bob can run the procedure $\alpha(n)$ times. Alice gets bits $b_{A,1}, b_{A,2} \ldots b_{A,\alpha(n)}$ and Bob gets $b_{B,1},b_{B,2}, \ldots, b_{B,\alpha(n)}$. Alice publishes $b_{A,1} \oplus b_{A,2}, \ldots, b_{A,1} \oplus b_{A,\alpha(n)}$ and Bob publishes $b_{B,1} \oplus b_{B,2}, \ldots, b_{B,1} \oplus b_{B,\alpha(n)}$. If they agree on over $1/2$ of the $\alpha(n)-1$ xors then they agree on $b_{A,1}$ and $b_{B,1}$. If $b_{A,1} \ne b_{B,1}$ the chance that over $1/2$ of the samples agree is $\leq \binom{\alpha(n)}{(\alpha(n)-1)/2}(1/6)^{(\alpha(n)-1)/2} \leq (2e/6)^{(\alpha(n)-1)/2}$. So if $\alpha(n) = \lg(n)^3$ we can bound their probability of disagreeing by $n^{\lg(n)}$.

Alice and Bob's running time increases by a factor of $\lg^3(n)$, but this subpolynomial so their running time remains  $\tilde{O}(\sqrt{s(N)}N^t + s(N)N^g)$. 

Consider the cyrptosystem from Lemma \ref{lem:keyExchangeHCBit}, if Alice and Bob wrote to the public ledger $I_T$ and the associated bit they agreed to was $b_I$ then there exists no $\PFT{T(n)\ell(n)}$ $A$ where $Pr_{I_T}[A(I_T,b_I)=True]-Pr_{I_t}[A(I_T,\overline{b_I})=True]>2f(n)$.  If there were we could run $A$ on $A(I_T,0)$ and $A(I_T,1)$, if they disagreed, return the bit in the procedure that returned true. If they both returned true of both returned false then return $0$ or $1$ uniformly at random. 

Now we note that an adversary can produce additional instances $\mathbb{I} = I'_1,b_{I'_1}, \ldots, I'_{\alpha(n)-1},b_{I'_{\alpha(n)-1}}$ knowing both $I'_i$ and $b_{I'_i}$. Then, the adversary can guess $b_I$ and produce the $\alpha(n)-1$ xor values. If a $\PFT{T(n)\ell(n)}$ algorithm $A$ exists 
$$Pr_{\mathbb{I},I_T}[A(\mathbb{I},I_T,b_{I'_1},\ldots, b_{I'_{\alpha(n)-1}})] -Pr_{\mathbb{I},I_T}[A(I_T,\overline{b_{I'_1}},\ldots,b_{I'_{\alpha(n)-1}})]< 2f(n)$$
then this would violate $\PFT{T(n)\ell(n)}$ $A'$ 
$$Pr_{I_T}[A'(I_T,b_I)] -Pr_{I_T}[A'(I_T,\overline{b_I})]< 2f(n).$$

So, no $\PFT{T(n)\ell(n)}$ algorithm $A$ exists such that 
$$Pr_{\mathbb{I},I_T}[A(\mathbb{I},I_T)=key]<1/2+1/f(n).$$	
\end{proof}